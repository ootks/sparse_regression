\section{Effeciently Computing $\eta_{T, k}.$}
\subsection{Linear Principal Minor Polynomials}
In this section, we will define a family of polynomials whose properties we will be considering in detail throughout this paper.

Let $\vec{a} = (a_S : S \in \binom{[n]}{k}) \in \R^{\binom{n}{k}}$, and let $X$ be a symmetric matrix of indeterminants.

We will associate to $\vec{a}$ the \emph{linear principal minor} (lpm) polynomial $p_{\vec{a}}(X)$,
\[
    p_{\vec{a}}(X) = \sum_{S \in \binom{[n]}{k} a_S\det(X|_S),
\]
where $X|_S$ denotes the principal submatrix of $X$ indexed by $S$.

One key family of examples of such polynomials are the \emph{characteristic coefficients}, which are defined as
\[
    c_k^n(X) = p_{\vec{1}](X) = \sum_{S \in \binom{[n]}{k} \det(X|_S).
\]
Here, $\vec{1}$ denotes the all 1's vector.

These arise naturally when considering the eigenvalues of a matrix, as we have the following formula for the characteristic polynomial of an $n\times n$ matrix:
\[
    \det(X + tI) = \sum_{k=0}^n c_{n-k}(X) t^k.
\]

In particular, we see that $c_k^n(X)$ invariant under change of basis, i.e. for any orthogonal matrix $U$, $c_k^n(X) = c_k^n(U^{\intercal}XU)$.

\subsection{Weighted Determinantal Point Processes}
We now let $\vec{a} = (a_S : S \in \binom{[n]}{k}) \in \R_+^{\binom{n}{k}}$, so that these coefficients are assumed to be nonnegative.
Let $M$ be a fixed positive semidefinite definite matrix.

We define the weighted determinantal point process for this set of coefficients to be a proability distribution on $S \in \binom{[n]}{k}$ with probability mass function given by
\[
    \mu_{M, \vec{a}}(S) \propto a_S \det(M|_S).
\]

To relate this to sparse linear regression, we can imagine using the following randomized variant of sparse linear regression.
Given the data $A$ and $b$, first draw a random subset $S$ according to $\mu_{AA^{\intercal}, \vec{a}}$, and then perform the regression using only the columns of $A$ in $S$.

We can then notate the expected error produced by this randomized linear regression.
\[
    \eta_{\vec{a}, T, k}(A,b) = \E_{\mu_{AA^{\intercal},\vec{a}}} [\ell(A_S, b) | T \subseteq S].
\]
Once again, this expectation serves as an upper bound 

We can then relate this quantity $\eta_{\vec{a}, T, k}(A, b)$ to the lpm polynomial $p_{\vec{a}}$:
\begin{theorem}
    If $AA^{\intercal}$ has rank at least $k$, then
\[
     \eta_{\vec{a}, T, k}(A,b) =
     \|b\|^2 - 
     \frac{p_{\vec{a}}(A^{\intercal}A + A^{\intercal}bb^{\intercal}A)}{p_{\vec{a}}(AA^{\intercal})} + 1
 \]
\end{theorem}
\begin{proof}
    We first recall the so-called matrix determinant lemma \cite{TODO}, which states that for any invertible $X \in \R^{n\times n }$ and $v \in \R^{n}$
    \[
        \det(X + vv^{\intercal}) = (1+v^{\intercal}X^{-1}v)\det(X).
    \]

    This implies that
    \begin{align*}
        p_{\vec{a}}(A^{\intercal}A + A^{\intercal}bb^{\intercal}A) &= \sum_{S \in \binom{[n]}{k}} a_S\det(A^{\intercal}A|_S + A^{\intercal}bb^{\intercal}A|_S)\\
                                                                   &= \sum_{S \in \binom{[n]}{k}} a_S\det(A^{\intercal}A|_S)\big(1+b^{\intercal}A_S(A^{\intercal}A|_S)^{-1}A_S^{\intercal}b\big)\\
    \end{align*}

    We also recall the closed form formula for $\ell(A,b)$, given by
    \[
        \ell(A,b) = \|b\|^2 - b^{\intercal}A(A^{\intercal}A)^{-1}A^{\intercal}b.
    \]

    We can thus simplify the above expression and see that

    \begin{align*}
        p_{\vec{a}}(A^{\intercal}A + A^{\intercal}bb^{\intercal}A) &= \sum_{S \in \binom{[n]}{k}} a_S\det(A^{\intercal}A|_S)\big(1+\|b\|^2-\ell(A_S,b)\big)\\
                                                                   &= (1+\|b\|^2)\left( \sum_{S \in \binom{[n]}{k}} a_S\det(A) \right) + \big(\sum_{S \in \binom{[n]}{k}}\det(A)\ell(A|_S,b)\big)\\
                                                                   &= (1+\|b\|^2)p_{\vec{a}}(A^{\intercal}A) + \big(\sum_{S \in \binom{[n]}{k}}\det(A^{\intercal}A|_S)\ell(A_S,b)\big)\\
    \end{align*}

    We then have that

    \begin{align*}
        (p_{\vec{a}}\frac{A^{\intercal}A + A^{\intercal}bb^{\intercal}A}{p_{\vec{a}}(A^{\intercal}A)} &= (1+\|b\|^2) + \big(\sum_{S \in \binom{[n]}{k}}\Pr(S)\ell(A_S,b)\big)\\
                                                                                                      &= \|b\|^2 - 
     \frac{p_{\vec{a}}(A^{\intercal}A + A^{\intercal}bb^{\intercal}A)}{p_{\vec{a}}(AA^{\intercal})} + 1
    \end{align*}
    
\end{proof}

