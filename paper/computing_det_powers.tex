\documentclass[a4paper]{article}
\usepackage[margin=.8in]{geometry}
\usepackage{graphicx}
\usepackage{fancyhdr, multicol}
\usepackage{float}
\fancyhead[L]{\LARGE{}}
\fancyhead[C] {\LARGE{Computing Sums of Powers of Determinants.}}
\fancyhead[R]{\LARGE{}}

\author{}
\title{Computing Sums of Powers of Determinants.}

%Set up fancy headers.
\pagestyle{fancy}
\usepackage{amsmath, amsthm, amsfonts, amssymb}
\usepackage{defs}
\usepackage[parfill]{parskip}

%Define theorem formatting
\newtheorem{theorem}{Theorem}
\newtheorem{lemma}{Lemma}

% Declare some common abbreviations
\newcommand{\R}{\mathbb{R}}
\newcommand{\C}{\mathbb{C}}
\newcommand{\N}{\mathbb{N}}
\newcommand{\Z}{\mathbb{Z}}
\newcommand{\Q}{\mathbb{Q}}
\newcommand{\E}{\mathbb{E}}
\newcommand{\Hom}{\textbf{\text{Hom}}}
\newcommand{\PP}{\textbf{P}}
\newcommand{\SPACE}{\textbf{SPACE}}
\newcommand{\NP}{\textbf{NP}}
\newcommand{\SAT}{\textbf{SAT}}
\newcommand{\pard}[2]{\frac{\partial #1}{\partial #2}}
\DeclareMathOperator*{\argmin}{arg\,min}
\DeclareMathOperator*{\argmax}{arg\,max}
\newcommand{\st}{{\text{ s.t. }}}
\newcommand{\ksets}{{\binom{[n]}{k}}}


\begin{document}
We seek to compute polynomials of the form
\[
    c_{n,k\ell} = \sum_{S \in \ksets} \det(X|_S)^{\ell}
\]

One way to compute this is to notice that $\det(X|_S)$ is the diagonal entry in the wedge power $\wedge^k X$ corresponding to the set $S$.

Therefore, we can equivalently think of this as computing
\[
    c_{n,k\ell} = \tr((\diag(\wedge^k X))^{\ell})
\]
Let $G$ be a subgroup of $\{-1, 1\}^n$ with the property that for any sets $A, B \in \ksets$ so that $A \neq B$, there is some element $x \in G$ so that 
\[
    \prod_{i \in A} x_i \neq 
    \prod_{i \in B} x_i.
\]

We then have that
\[
    \diag(\wedge^k X) = \sum_{x \in G} \wedge^k (D_xXD_x),
\]
where $D_x$ is the sign matrix where $(D_x)_{ii} = x_i$.

We then have that 
\begin{align*}
    c_{n,k\ell} &= \tr((\diag(\wedge^k X))^{\ell})\\
                &= \sum_{x_1 ,\dots, x_{\ell} \in G} \tr\left(\wedge^k \left(\prod_{i=1}^{\ell}D_{x_i}XD_{x_i}  \right) \right)\\
                &= \sum_{x_1 ,\dots, x_{\ell} \in G} c_{n,k}\left(\prod_{i=1}^{\ell}D_{x_i}XD_{x_i}\right)
\end{align*}
Each term in the sum requires $O(\poly(n,k,\ell))$  time to compute, and there are $|G|^{\ell}$ terms to compute.

So, how small can we make the group $G$? Can it be made of size $\poly(n, k)$? Then at least for fixed $\ell$, this computation can be done in polynomial time.
\end{document}
