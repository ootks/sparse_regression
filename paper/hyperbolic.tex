\section{Conncection to Hyperbolic Optimization}
A multivariate polynomial $p \in \R[x_1 ,\dots, x_n]$ is said to be \emph{hyperbolic} with respect to fixed $e \in \R^n$ if $p(e) > 0$, and for all $x \in \R^n$, the univariate polynomial $g(t) = p(x+te)$ is real rooted, in the sense that the only complex numbers $t$ where $g(t) = 0$ are real.

Hyperbolic polynomials were originally defined in the work of Garding in differential equations \cite{TODO}, and theey have been the subject of much study because of their connections to combinatorics and optimization \cite{TODO}.

One object of particular interest attached to a polynomial $p$ is the \emph{hyperbolicity cone} of $p$ with respect to $e$.

If $p$ is hyperbolic with respsect to $e$, then the hyperbolicity cone of $p$ with respect to $e$ is defined as
\[
    \Lambda_e(p) = \{x \in \R^n : \forall t \ge 0, p(x+te) \ge 0\}.
\]

The hyperbolicity cone of a polynomial is known to be convex, and moreover, it is known that the function $\log(p(x))$ is a self-concordant barrier function for $\Lambda_e(p)$ \cite{TODO}. 
For our purposes, it is sufficient to note that $p(x) > 0$ for $x \in \relint\Lambda_e(p)$, and $p(x) = 0$ for $x \in \partial \Lambda_e(p)$.

An important example of a hyperbolic polynomial is the determinant of a symmetric matrix, which is hyperbolic with respect to the identity matrix. Hyperbolicity in this case follows immediately from the spectral theorem. The associated hyperbolicity cone is precisely the positive semidefinite cone.

The main focus of this paper 
