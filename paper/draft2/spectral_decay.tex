\documentclass[a4paper]{article}
\usepackage[margin=.8in]{geometry}
\usepackage{graphicx}
\usepackage{fancyhdr, multicol}
\usepackage{algorithm}
\usepackage{algpseudocode}
\usepackage{float}
\fancyhead[L]{\LARGE{Kevin Shu}}
\fancyhead[C] {\LARGE{Sparse Linear Regression with Spectral Decay}}
\fancyhead[R]{\LARGE{}}

\author{Kevin Shu}
\title{Sparse Linear Regression with Hyperbolic Optimization}

%Set up fancy headers.
\pagestyle{fancy}
\usepackage{amsmath, amsthm, amsfonts, amssymb}
\usepackage[parfill]{parskip}

%Define theorem formatting
\newtheorem{theorem}{Theorem}
\newtheorem{lemma}{Lemma}

% Declare some common abbreviations
\newcommand{\R}{\mathbb{R}}
\newcommand{\C}{\mathbb{C}}
\newcommand{\N}{\mathbb{N}}
\newcommand{\Z}{\mathbb{Z}}
\newcommand{\Q}{\mathbb{Q}}
\newcommand{\E}{\mathbb{E}}
\newcommand{\ksets}{\binom{[n]}{k}}
\newcommand{\Hom}{\textbf{\text{Hom}}}
\newcommand{\PP}{\textbf{P}}
\newcommand{\SPACE}{\textbf{SPACE}}
\newcommand{\NP}{\textbf{NP}}
\newcommand{\SAT}{\textbf{SAT}}
\newcommand{\pard}[2]{\frac{\partial #1}{\partial #2}}
\DeclareMathOperator*{\argmin}{arg\,min}
\DeclareMathOperator*{\tr}{tr}
\DeclareMathOperator*{\supp}{supp}
\DeclareMathOperator*{\argmax}{arg\,max}
\newcommand{\st}{{\text{ s.t. }}}

\begin{document}
The sparse linear regression problem is
\[
    \min \|A x - b\|^2 \text{ where }\|x\|_0 \le k
\]
This is equivalent to the following conic optimization problem:
\begin{equation}\label{eq:tight}
    \alpha_k(A, b) = 
    \|b\|^2 - \min \{y : A^{\intercal}A y - A^{\intercal}bb^{\intercal}A \in S^{n}_k\}
\end{equation}
We define the hyperbolic relaxation
\begin{equation}\label{eq:relax}
    \eta_k(A, b) = 
    \|b\|^2 - \min \{y : A^{\intercal}A y - A^{\intercal}bb^{\intercal}A \in H^{n}_k\}
\end{equation}
\section{Summary of Results}
We can find a closed form for $\eta_k(A,b)$.
\begin{lemma}
    \[
        \eta_k(A, b) = \|b\|^2 - \frac{b^{\intercal} A\nabla c_k^n(A^{\intercal}A)A^{\intercal} A^{\intercal}b}{c_k^n(A^{\intercal}A)}.
    \]
\end{lemma}

We can find an exact expression for $\alpha_k$ in terms of diagonal rescaling and $\eta_k$.
\begin{lemma}
\[\alpha_k(A, b) =  \min \{ \eta_k(AD, b) : {D \text{ is PSD and diagonal}}\}.\]
\end{lemma}

We can compute $\eta_k(A,b)$ efficiently.
\begin{lemma}
    The quantity $\eta_k(A, b)$ can be computed in $O(kn^{\omega})$ time.
\end{lemma}

We can find a subset whose linear regression error matches this upper bound bound efficiently.
\begin{lemma}
    We can find $S \subseteq [n]$ with $\alpha_k(A|_S, b) \ge \eta_k(A, b)$ in at most $O(k^2n^{1+\omega})$ time.
\end{lemma}

We can recover a primal solution from a dual solution.
\begin{lemma}
    Let $y^*$ be a minimizer in 
\[
    \min \{y : A^{\intercal}A y - A^{\intercal}bb^{\intercal}A \in H^n_k\},
\]
and let $Y^* = A^{\intercal}A y^* - A^{\intercal}bb^{\intercal}A$.

Then $X^* = \frac{\nabla c_k^n(Y)}{\tr(AA^{\intercal}\nabla c_k^n(Y))}$ is an optimal solution for the primal problem
\[
    \max \{\tr(A^{\intercal}bb^{\intercal}A X) : \tr(AA^{\intercal} X) = 1, \; X \in H^n_k^*\}.
\]

Given the value of $y^*$, $X^*$ can be computed in $O(n^{\omega} + n^2k)$ time.
\end{lemma}


We will say that $A$ exhibits $(\epsilon, k)$-spectral decay if $\epsilon \sigma_{k}(A) \ge \sigma_{k+1}(A)$.
Let $\ell_k(A,b)$ be the error incurred by regressing $b$ against the top $k$ singular vectors of $A$. We define these terms precisely below.

\begin{lemma}
    If $A$ exhibits $(\epsilon, k)$ spectral decay, then for all $b$, we have that 
    \[
        \alpha_k(A, b) \le (1-\frac{1}{(1+\epsilon^2)^{nk}}) \|b\|^2 + \frac{1}{(1+\epsilon^2)^{nk}}\ell_k(A, b).
    \]
\end{lemma}


\section{Closed Form for Hyperbolic Optimization}
Much in the same way that we have a closed form solution for the optimizer for linear regression, we can actually obtain a closed form solution for $\eta_k(A, b)$ in terms of the characteristic coefficient $c_k^n(A^{\intercal}A)$.
\begin{lemma}
    \[
        \eta_k(A, b) = \|b\|^2 - \frac{b^{\intercal} A\nabla c_k^n(A^{\intercal}A)A^{\intercal} A^{\intercal}b}{c_k^n(A^{\intercal}A)}.
    \]
\end{lemma}
\begin{proof}
    For convenience, let $v = A^{\intercal}b$.

    It is clear that at the smallest possible $y$ so that $A^{\intercal}Ay - vv^{\intercal} \in H^n_k$, we have that 
    \[
        c_k^n(A^{\intercal}Ay - vv^{\intercal}) = 0.
    \]
    The ``matrix determinant lemma'', which states that $\det(X - vv^{\intercal}) = (1-v^{\intercal}X^{-1}v)\det(X)$, immediately implies
    \begin{align*}
        c_k^n(A^{\intercal}Ay - vv^{\intercal}) &= \sum_{S \subseteq [n] : |S| = k}(1-\frac{v^{\intercal}|_S(A^{\intercal}A)|_S^{-1}v|_S}{y})\det(A^{\intercal}A|_S)y^k\\
                                              &= c_k^n(A^{\intercal}A)y^k - v^{\intercal}\nabla c_k^n(A^{\intercal}A) vy^{k-1}
    \end{align*}
    The only nonzero root of this polynomial is
    \[
        y = \frac{v^{\intercal}\nabla c_k^n(A^{\intercal}A) v}{c_k^n(A^{\intercal}A)}
    \]

    This must then be the minimizer of this expression, and so 
    \[
        \eta_k(A, b) = \|b\|^2 - y
    \]
\end{proof}
Notice that when $k = n$, this is equivalent to the closed form for the linear regression objective.

As a note, we can upgrade this result to an exact equality using diagonal congruence.
\begin{lemma}
\[\alpha_k(A, b) =  \min \{ \eta_k(AD, b) : {D \text{ is PSD and diagonal}}\}.\]
\end{lemma}
\begin{proof}
    Let $S \subseteq [n]$ so that $S = \supp x^*$ where $x^*$ is the optimal regressor in the sparse linear regression problem, and $\supp x^*$ denotes the set of nonzero entries of $x^*$. 

    We then have for the closed form for the linear regression problem that $x^* = (A^{\intercal}A|_S)^{-1}(A|_Sb|_S)$.

    Let $D$ be the diagonal matrix so that  $D_{ii} = 1_{i \in S}$. Then $c_k^n(DA) = \det(A|_S)$, and
    \begin{align*}
        \|b\|^2 - \alpha_k(AD, b) &= \|b\|^2 - \frac{b^{\intercal}A|_S\nabla c_k^n(A^{\intercal}A|_S) A|_S^{\intercal}b}{c_k^n(A^{\intercal}A|_S)}\\
                                                                                                                            &= \|b\|^2 - (A|_S^{\intercal}b)^{\intercal}(A^{\intercal}A|_S)^{-1} (A|_S^{\intercal}b)\\
                                                                                                                            &= \|b\|^2 - x^*A^{\intercal}Ax^*,
    \end{align*}
    which is the linear regression error for $x^*$.
\end{proof}
\subsection{Connection to Determinantal Point Processes}
We consider a random subset $S \subseteq pn[$ chosen so that $|S| = k$,
\[
    \Pr(S) \propto \det(A^{\intercal}A|_S).
\]
This is known as a truncated determinantal point process, and it is known that it is possible to sample from this distribution using Markov Chain Monte Carlo techniques.

\begin{lemma}
    \[
        \eta_k(A, b) = \E[\alpha_k(A|_S, b)].
    \]
\end{lemma}
\begin{proof}
    We have that for any set $S$, the probability of set $S$ appearing is precisely
    \[
        \frac{\det(A^{\intercal}A|_S)}{\sum_{S \in \ksets} \det(A^{\intercal}A|_S)} = \frac{\det(A^{\intercal}A|_S))}{c_k^n(A^{\intercal}A)}
    \]
    We also have from the closed form for the linear regression problem that
    \[
        \alpha_k(A|_S,b) = \|b\|^2 - b^{\intercal}A|_S(A^{\intercal}A)|_S^{-1}A|_S^{\intercal}b.
    \]
    Therefore,
    \begin{align*}
        \E[\alpha_k(A|_S,b)] &= \|b\|^2 - \sum_{S \in \ksets} \frac{\det(A^{\intercal}A)}{c_k^n(A^{\intercal}A)}A|_Sb^{\intercal}A|_S(A^{\intercal}A)|_S^{-1}A|_S^{\intercal}b\\
                             &= \|b\|^2 - b^{\intercal}A|_S\left( \sum_{S \in \ksets} \frac{\nabla \det(A^{\intercal}A|_S)}{c_k^n(A^{\intercal}A)} \right)A|_S^{\intercal}b\\
                             &= \|b\|^2 - \frac{b^{\intercal}\nabla c_k^n(A^{\intercal}A|_S)A|_S^{\intercal}b}{c_k^n(A^{\intercal}A)}
    \end{align*}



    Therefore,
    \[
        \alpha_k(A|_S,b) = \|b\|^2 - b^{\intercal}A|_S(A^{\intercal}A)|_S^{-1}A|_S^{\intercal}b
    \]
\end{proof}

\section{Algorithmic Questions}
\begin{lemma}
    The quantity $\eta_k(A, b)$ can be computed in $O(kn^{\omega})$ time.
\end{lemma}
\begin{proof}
    Recall that 
    \[
        c_k^n(A^{\intercal}Ay - vv^{\intercal}) = c_k^n(A^{\intercal}A)y^k - v^{\intercal}\nabla c_k^n(A^{\intercal}A) v y^{k-1}
    \]
    We first claim that given a matrix $X$, we can compute $c_k^n(X)$ in $kn^{\omega}$ time.

    To compute $c_k^n(X)$, we can apply the Faddeev–LeVerrier algorithm, which states that
    \[
        c_k^n(X) = \frac{(-1)^{k}}{k!} \det(Y),
    \]
    where $Y$ is the $k\times k$ matrix given by 
    \[
        Y =
        \begin{pmatrix}
            \tr A & m-1 & 0 & \dots\\
            \tr A^2 & \tr A & m - 2 & 0 & \dots\\
            \vdots & \vdots & & & \vdots\\
            \tr A^{k-1} & \tr A^{k-2} & \dots & \dots & 1\\
            \tr A^k & \tr A^{m-1} & \dots & \dots & \tr A
        \end{pmatrix}.
    \]

    The $A^i$ can be computed using $O(n^{\omega})$ operations each, and the determinant of $Y$ can be computed in $O(k^{\omega})$ time.

    Now, it is clear that we can compute $c_k^n(A^{\intercal}A)$ in $O(kn^{\omega})$ time, and also
    \[
        v^{\intercal}\nabla c_k^n(A^{\intercal}A) v = c_k^n(A^{\intercal}A) - c_k^n(A^{\intercal}A - vv^{\intercal})
    \]
    So, we can compute these two quantities in $O(kn^{\omega})$ time, and then compute their ratio in $O(1)$ time.
\end{proof}
\begin{remark}
    The Fadeev-Leverrier algorithm is famously numerically unstable, though for small $k$, its performance is acceptable. On the other hand, it is also possible to compute the $c_k^n(X)$ in a numerically stable way by noting that 
    \[
        \det(X + t I) = \sum_{i=0}^n c_k^n(X)t^i.
    \]
    By evaluating the determinant of $X+tI$ at $n$ distinct values of $t$, and then interpolating to find the coefficients of this characteristic polynomial, we are able to compute the characteristic polynomial in $n^{\omega+1}$ time in a numerically stable way.
\end{remark}
\subsection{Finding A Primal Solution}
If we have an optimum for the dual problem
\[
    \min \{y : A^{\intercal}A y - A^{\intercal}bb^{\intercal}A \in H^n_k\},
\]
how can we extract from this a primal optimum solution? 

\begin{lemma}
    Let $y^*$ be a minimizer in 
\[
    \min \{y : A^{\intercal}A y - A^{\intercal}bb^{\intercal}A \in H^n_k\},
\]
and let $Y^* = A^{\intercal}A y^* - A^{\intercal}bb^{\intercal}A$.

Then $X^* = \frac{\nabla c_k^n(Y)}{\tr(AA^{\intercal}\nabla c_k^n(Y))}$ is an optimal solution for the primal problem
\[
    \max \{\tr(A^{\intercal}bb^{\intercal}A X) : \tr(AA^{\intercal} X) = 1, \; X \in H^n_k^*\}.
\]
Given the value of $y^*$, $X^*$ can be computed in $O(n^{\omega} + n^2k)$ time.
\end{lemma}
\begin{proof}
    It is clear that $A^{\intercal}A y - A^{\intercal}bb^{\intercal}A \in \parital H^n_k$, and it is a general fact that if $Y \in \partial H^n_k$, then $\nabla c_k^n(Y) \in H^n_k^*$, and $\langle \nabla c_k^n(Y), Y \rangle = 0$.
    Also, by our choice of normalization, $X^*$ also satisfies $\tr(AA^{\intercal}X^*) = 1$, so it is feasible.

    Hence, by complementary slackness, $X^*$ is a primal optimal.

    If we let $Y^* = UDU^{\intercal}$ be the diagonalization of $Y^*$, then $\nabla c_k^n(Y^*) = U\nabla c_k^n(D)U^{\intercal}$.
    The diagonalization of $Y^*$ can be computed in $O(n^{\omega})$ time.
    
    $\nabla c_k^n(D) $ is a diagonal matrix, where
    \[
        \nabla c_k^n(D)_{ii} = e_{k-1}^{n-1}(D_{11}, \dots, \hat{D}_{ii}, \dots, D_{nn}),
    \]
    where $\hat{D}_{ii}$ denotes the fact that $D_{ii}$ is removed from this list of arguments.
    
    Each $e_{k-1}^{n-1}$ can be computed in $O(nk)$ time for each coefficient, so that the total time required to compute $\nabla c_k^n(D)$ in $n^2k$ time.

\end{proof}
Now, we know that $H^n_k^* \subseteq FW^n_k$, so that there are $x_1, \dots, x_m$ so that $\|x_i\|_0 \le k$, and
\[
    X^* = \sum_{i=1}^m x_ix_i^{\intercal}.
\]
By general convex geometry, there must be some $i$ so that $x_i$ acheives a small primal objective.

While this guarantees a theoretical lower bound on the quality of some sparse solution, we do not know how to find this decomposition computationally.

\subsection{Polytime algorithm for getting sparse regressor}
In practice, heurstic methods for finding this decomposition of $X$ may lead to fast algorithms.
Here, we give a polytime algorithm for finding a set $S$ that matches the lower bound given by the hyperbolic relaxation, that is, so that $\alpha_k(A|_S, b) \ge \eta_k(A, b)$.

To do this, we need to define some more hyperbolic polynomials. For $T \subseteq S$ with  $|T| \le k$, let
\[
    c_{T,k}^n(X) = \sum_{S\in \ksets : T \subseteq S} \det(X|_S).
\]
In particular, $c_{\varnothing, k}^n = c_k^n$, and $c_{T, |T|}^n = \det(X|_T)$.

It is a general fact that $c_{T,k}^n$ is hyperbolic with respect to any PSD matrix.
So, we can let $H_{T, k}^n = \Lambda(c_{T,k}^n)$ be the hyperbolicity cone of $c_{T,k}^n$ containing the PSD cone.

As above, we let $\eta_{T, k}(A,b) = \min \{y : A^{\intercal}A y - A^{\intercal}bb^{\intercal}A \in H^{T,n}_k\}$.

We can actually reproduce most of the results above about $c_k^n$ in this more general setting.

\begin{lemma}
    \[\eta_{T,k}(A,b) = \|b\|^2 - \frac{b^{\intercal}A\nabla c_{T,k}^n(A^{\intercal}A)A^{\intercal}b}{c_{T,k}^n(A^{\intercal}A)}.\]
\end{{lemma}
\begin{proof}
    As above, we know that at an optimum for $\eta_{T,k}(A,b)$
    \[c_{T,k}^n(A^{\intercal}A y - A^{\intercal}bb^{\intercal}A) = 0.\]

    Because $A^{\intercal}bb^{\intercal}A$ is rank 1, this polynomial still has a zero of order $k-1$ at 0, and the result follows from examining the two nonzero coefficients of this polynomial.
\end{proof}

\begin{lemma}
    The quantity $\eta_{T,k}(A, b)$ can be computed in $O(kn^{\omega})$ time.
\end{{lemma}
\begin{proof}
    As above, it suffices to compute $c_{T,k}^n(X)$ in this much time.

    To do this, we use the Schur complement lemma that if $T \subseteq S$, then
    \[
        \det(X|_S) = \det(X|_T) \det((X \setminus T)|_{S\setminus T}),
    \]
    where $X \setminus T$ is the Schur complement 
    \[
        X|_{[n]-T} - X|_{T, [n]-T}X|_T^{-1}X|_{[n]-T, T}.
    \]

    We see then that 
    \[
        c_{T,k}^n(X) = \det(X|_T) c_{k-|T|}^n(X \setminus T).
    \]
    This can be computed in $O(kn^{\omega})$ time.
\end{proof}
\begin{lemma}
    For any $T \subseteq [n]$ with $|T| \le k$, there is some $i \not \in T$ so that 
    \[
        \eta_{T, k}(A,b) \ge \eta_{T+i, k}(A,b)
    \]
\end{lemma}
\begin{proof}

    These polynomials satisfy the relation that 
    \[
        c_{T,k}^n(X) = \frac{1}{n-|T|} \sum_{i \not \in T} c_{T+i, k}^n(X).
    \]

    Therefore, if for some $y$,
    \[
        c_{T,k}^n(A^{\intercal}A y - A^{\intercal}bb^{\intercal}A) = 0.
    \]

    Then there must be some $i \not \in T$ so that 
    \[
        c_{T+i,k}^n(A^{\intercal}A y - A^{\intercal}bb^{\intercal}A) \le 0.
    \]
    Now, for this choice of $i$, consider
    \[
        \lim_{y \rightarrow \infty} c_{T+i,k}^n(A^{\intercal}A y - A^{\intercal}bb^{\intercal}A) = \infty.
    \]
    Therefore, by the intermediate value theorem, for some $y' \ge y$, we must have
    \[
        c_{T+i,k}^n(A^{\intercal}A y' - A^{\intercal}bb^{\intercal}A) = 0.
    \]
    We have seen that this polynomial $c_{T+i,k}^n$ has at most one nonzero root, and so, we see that 
    \[
        \eta_{T, k}(A,b) \ge \eta_{T+i, k}(A,b).
    \]
\end{proof}
\begin{lemma}
    We can find $S \subseteq [n]$ with $\alpha_k(A|_S, b) \ge \eta_k(A, b)$ in at most $O(k^2n^{1+\omega})$ time.
\end{lemma}
\begin{proof}
    We do this iteratively. We maintain a set $T$, which starts as $\varnothing$, and we will maintain the invariant that $\eta_{T, k}(A, b) \ge \eta_k(A,b)$.

    We then procede with rounds of computation.

    If at the start of a round, $|T| = k$, then we output $T$. Clearly, this output will be correct, since
    \[
        \alpha_k(A|_T, b) = \eta_{T, k}(A,b) \ge \eta_k(A,b).
    \]


    If $|T| < k$, we continue the round by computing
    \[
        \eta_{T+i, k}(A,b)
    \]
    for each $i \in [n] \setminus T$. This takes $O(kn^{1+\omega})$ time per round.

    By the previous lemma, we know that there must be some $i$ so that
    \[
        \eta_{T+i, k}(A,b) \ge \eta_{T,k}(A,b) \ge \eta_k(A,b),
    \]
    and that implies that at the end of each round, we will have found some $i$ so that 
    \[
        \eta_{T+i, k}(A,b) \ge \eta_k(A,b).
    \]
    We replace $T$ by $T+i$, and procede to the next round of computations.

    We perform at most $k$ rounds of computations, and each round requires $O(kn^{1+\omega})$ time, which implies the claim.
\end{proof}


This procedure suggests a greedy approach to sparse linear regression.
\begin{algorithm}
    \caption{A Greedy Algorithm for Sparse Linear Regression}
    \begin{algorithmic}\label{alg:greedy}
        \State $T \gets \varnothing$
        \For{$t = 1 \dots k$}
            \State $j \gets \argmax \eta_{T+j, k}(A,b)$
            \State $T \gets T + j$
        \EndFor

        \Return T
    \end{algorithmic}
\end{algorithm}
Presumably this can be incorporated into some kind of branch and bound framework.


\section{Sparse Linear Regression with Spectral Decay}
Let the singular value decomposition of $A$ be 
\[
    A = U\Sigma V
\]
where $U$ and $V$ are orthogonal, and $\Sigma$ is diagonal, where $\Sigma_{ii}  = \sigma_i(A)$, and $\sigma_1(A) \ge \sigma_2(A) \dots $.

The PCA problem is to find the $k$ largest singular values of $A$, and the corresponding $k$ singular directions, $v_1, \dots, v_k$.
A common property of real world data matrices is that $\sigma_i(A)$ is very small for $i$ much smaller than $n$, so that $A$ is actually well approximated by the first $k < n$ singular vectors.

We define the spectral sparsification of $A$ to be 
\[
    A^{(k)} = U\Sigma^{(k)}V
\]
Here, $\Sigma^{(k)}$ is obtained by setting all but the first $k$ columns of $\Sigma$ to 0.

For $b \in \R^m$, we define
\[
    \ell_k(A, b) = \min \{ \|b\|^2 - x^{\intercal} A^{(k)}^{\intercal}b b^{\intercal}A^{(k)}x : x \in \R^n\} =  \|b\|^2 - (b^{\intercal} U)|_{[k]}(U^{\intercal}b)|_{[k]}
\]
That is, we use the spectral sparsification of $A$ to perform linear regression on $b$.

We will say that $A$ exhibits $(\epsilon, k)$-spectral decay if $\epsilon \sigma_{k}(A) \ge \sigma_{k+1}(A)$.

\begin{lemma}
    If $A$ exhibits $(\epsilon, k)$ spectral decay, then for all $b$, we have that 
    \[
        \alpha_k(A, b) \le (1-\frac{1}{(1+\epsilon^2)^{nk}}) \|b\|^2 + \frac{1}{(1+\epsilon^2)^{nk}}\ell_k(A, b).
    \]
\end{lemma}
\begin{proof}
    We recall that 
    \[
        \alpha \le \|b\|^2 - \frac{(Ab)^{\intercal} \nabla c_k(A^{\intercal}A) (Ab)}{c_k(A^{\intercal}A)}.
    \]
    Recall that $c_k$ is basis invariant, so that 
    \begin{align*}
        \frac{b^{\intercal}A \nabla c_k(A^{\intercal}A) A^{\intercal}b}{c_k(A^{\intercal}A)} &= 
        \frac{(U\Sigma V b)^{\intercal} \nabla c_k(V^{\intercal}\Sigma^2V) (AU\Sigma Vb)}{c_k^n(\Sigma^2)}\\
                                                                                     &=\frac{b^{\intercal} U\Sigma \nabla c_k^n(\Sigma^2) \Sigma U^{\intercal}b}{c_k^n(\Sigma^2)}
    \end{align*}

    \[
        \nabla c_k^n(\Sigma^2) = \sum_{S \in \ksets} \det(\Sigma|_S)^2 \Sigma|_S^{-2}
    \]
    Now, $\Sigma|_S^2$ is PSD for every $S \in \ksets$, we see that 
    \[
        \nabla c_k^n(\Sigma^2) \succeq \det(\Sigma|_{[k]})^2 \Sigma|_{[k]}^{-2},
    \]
    and so we obtain 

    \[
        \frac{b^{\intercal}A \nabla c_k^n(A^{\intercal}A) A^{\intercal}b}{c_k^n(A^{\intercal}A)} \ge 
        \frac{\det(\Sigma|_{[k]})^2}{c_k^n(\Sigma^2)}(b^{\intercal}U)|_{[k]}(U^{\intercal}b)|_{[k]} = 
        \frac{\det(\Sigma|_{[k]})^2}{c_k^n(\Sigma^2)}(\|b\|^2 - \ell_k(A,b)).
    \]
    To show the result, it remains to show that if $A$ has $(\epsilon, k)$ spectral decay, then
    \[
        \frac{\det(\Sigma|_{[k]})^2}{c_k^n(\Sigma^2)} \le \frac{1}{(1+\epsilon)^{nk}},
    \]
    or equivalently,
    \[
        \frac{c_k^n(\Sigma^2)}{\det(\Sigma|_{[k]})^2} \le (1+\epsilon)^{nk},
    \]
    We expand $c_k^n$ out as a sum over $S \in \ksets$, and reindex the summation to be over $|S \Delta [k]|$ first.
    \begin{align*}
        c_k^n(\Sigma^2) = \sum_{\ell = 0}^k\sum_{S \in \ksets : |S \Delta [k]| = \ell} \prod_{i\in S}\sigma_i^2
    \end{align*}

    Now, notice that if $|S\Delta [k]| = \ell$, then $\prod_{i\in S}\sigma_i^2 \le \epsilon^{2\ell}\prod_{i=1}^k \sigma_i^2$.
    There are $\binom{k}{\ell} \binom{n}{\ell}$ sets with the property that $|S\Delta [k]|$, so that 
    \[
        c_k^n(\Sigma^2) \le \prod_{i=1}^k \sigma_i^2 \left(\sum_{\ell=0}^k\epsilon^{2\ell}\binom{k}{\ell} \binom{n}{\ell}\right)
    \]

    Now, we note that 
    \[
        \binom{k}{\ell} \binom{n}{\ell} \le \binom{nk}{\ell},
    \]
    which follows from simply expanding out both sides.

    Therefore, 
    \[
        c_k^n(\Sigma^2) \le \prod_{i=1}^k \sigma_i^2 \left(\sum_{\ell=0}^{nk}\epsilon^{2\ell}\binom{nk}{\ell}\right) = \det(\Sigma|_{[k]}^2) (1+\epsilon^2)^{nk}.
    \]
    This gives the desired result. 
\end{proof}

\end{document}
