\documentclass[a4paper]{article}
\usepackage[margin=.8in]{geometry}
\usepackage{graphicx}
\usepackage{fancyhdr, multicol}
\usepackage{float}
\fancyhead[L]{\LARGE{}}
\fancyhead[C] {\LARGE{Sparse PCA and Hyperbolic Optimization}}
\fancyhead[R]{\LARGE{}}

\author{}
\title{Sparse PCA and Hyperbolic Optimization}

%Set up fancy headers.
\pagestyle{fancy}
\usepackage{amsmath, amsthm, amsfonts, amssymb}
\usepackage[parfill]{parskip}
\usepackage{defs}

%Define theorem formatting
\newtheorem{theorem}{Theorem}
\newtheorem{lemma}{Lemma}

\begin{document}
The sparse PCA problem is given by 
\begin{equation}\label{eq:orig}
\begin{aligned}
    \text{maximize} &&x^{\intercal}Ax\\
    \text{such that } &&\|x\|_2 = 1\\
                      &&\|x\|_0 \le k
\end{aligned}
\end{equation}
Here, $\|x\|_0$ denotes the number of nonzero entries of $x$, and we take $A$ to be PSD.

We can reformulate this in terms of the factor-width-$k$ cone:
\begin{equation}
\begin{aligned}
    \text{maximize} &&\tr(AX)\\
    \text{such that } &&\tr(X) = 1\\
                      &&X \in \FW^n_k
\end{aligned}
\end{equation}
Here, $\FW^n_k = \{ xx^{\intercal} : \|x\|_0 \le k\}$.

The dual to this problem is 
\begin{equation}\label{eq:dual}
\begin{aligned}
    \text{minimize} &&y\\
    \text{such that } &&Iy - A \in S^n_k
\end{aligned}
\end{equation}
where $S^n_k = \{X \in \Sym(\R^n) : \forall S \in \binom{[n]}{k}, X|_S \succeq 0\}$.
In fact, by Slater's condition, this is equivalent to the primal problem.

For any homogeneous hyperbolic LPM polynomial $p$ of degree $k$, we have that $S^n_k \subseteq \Lambda^+_I(p)$, so we obtain a lower bound on Equation \ref{eq:dual}:
\begin{equation}\label{eq:hyp}
\begin{aligned}
    \text{minimize} &&y\\
    \text{such that } &&Iy - A \in \Lambda^+_I(p)
\end{aligned}
\end{equation}

The value of Equation \ref{eq:hyp} is precisely the maximum hyperbolic eigenvalue of $A$ with respect to $p$ in the $I$ direction, which we denote by $\lambda_{max}(A)$. This follows because 
\[
    \lambda_{max}(A)I - A \in \Lambda^+_I(p),
\]
and for any $y < \lambda_{max}(A)$, this fails.

To conclude, we see that $\lambda_{max}(A)$ is a lower bound on the sparse PCA problem for any hyperbolic LPM polynomial $p$. 

On the other hand, if we let $x^&$ be the optimal value of \ref{eq:orig}, and $S$ be the support of $x^*$, then for the hyperbolic polynomial $p(X) = \det(X|_S)$, $\lambda_{max}(A)$ is precisely the value of the sparse PCA problem.
\end{document}
