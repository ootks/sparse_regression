\section{Effeciently Computing $\eta_{T, k}.$}
We will be making use of a number of theorems from linear algebra, such as the matrix-determinant lemma for rank 1 updates to the determinant and the Schur complement lemma.
All of these results can be found in any standard reference on linear algebra, and we recommend \cite{horn2012matrix}.
\subsection{Linear Principal Minor Polynomials}
In this section, we will define a family of polynomials whose properties we will be considering in detail throughout this paper.

Let $\vec{a} = (a_S : S \in \binom{[n]}{k}) \in \R^{\binom{n}{k}}$, and let $X$ be a symmetric matrix of indeterminants.

We will associate to $\vec{a}$ the \emph{linear principal minor} (lpm) polynomial $p_{\vec{a}}(X)$,
\[
    p_{\vec{a}}(X) = \sum_{S \in \binom{[n]}{k}} a_S\det(X|_S),
\]
where $X|_S$ denotes the principal submatrix of $X$ indexed by $S$.

One key family of examples of such polynomials are the \emph{characteristic coefficients}, which are defined as
\[
    c_k^n(X) = p_{\vec{1}}(X) = \sum_{S \in \binom{[n]}{k}} \det(X|_S).
\]
Here, $\vec{1}$ denotes the all 1's vector.

These arise naturally when considering the eigenvalues of a matrix, as we have the following formula for the characteristic polynomial of an $n\times n$ matrix:
\[
    \det(X + tI) = \sum_{k=0}^n c_{n-k}^n(X) t^k.
\]

In particular, we see that $c_k^n(X)$ invariant under change of basis, i.e. for any orthogonal matrix $U$, $c_k^n(X) = c_k^n(U^{\intercal}XU)$.

One operation that we will need to perform on these polynomials is what we call \emph{conditioning}, which we will use in the next subsection to compute conditional expectations for weighted determinantal point processes.

Given a lpm polynomial $p_{\vec{a}}$ of degree $k$, we define the conditioning of $p_{\vec{a}}$ on a set $T \subseteq [n]$ with $|T| \le k$ by
\[
    p_{\vec{a}, T}(X) = \sum_{S \in \binom{[n]}{k} : T \subseteq S} a_S \det(X|_S)
\]
We will want a formula for $p_{\vec{a}, T}$ in terms of $p_{\vec{a}}$.
To do this, we will need to define the Schur complement of $X$ when $X$ is a symmetric matrix, and $T \subseteq [n]$. Suppose we have the following block structure for $X$,
\[
    X = 
    \begin{pmatrix}
        X|_T & X_{T, [n] \setminus T}\\
        X_{[n] \setminus T, T} & X|_{[n] \setminus T}\\
    \end{pmatrix}
\]
then the Schur complement of $X$ with respect to $T$ is
\[
    X \setminus T = X|_{[n]  - T} -  X_{[n] \setminus T, T} X|_{T}^{-1}X_{T, [n] \setminus T}.
\]

\begin{lemma}\label{lem:poly_cond}
\[
    p_{\vec{a}, T}(X) = \det(X|_T) \left( \sum_{i \in T} \frac{\partial}{\partial X_{ii}} \right)^{|T|}p_{\vec{a}}(X \setminus T) 
\]
\end{lemma}
\begin{proof}
    Firstly, it is possible to see from the polynomial expansion of the determinant that
    \[
        \frac{\partial}{\partial X_{ii}}\det(X|_S) = 
        \begin{cases}
            0 \text{ if }i \not \in S\\
            \det(X|_{S \setminus i}) \text{ otherwise}.            
        \end{cases}
    \]
    Therefore, we can see by induction that 
    \[
        \left( \sum_{i \in T} \frac{\partial}{\partial X_{ii}} \right)^{|T|}\det(X|_S) = 
        \begin{cases}
            0 \text{ if }T\not \subseteq S\\
            \det(X|_{S \setminus T}) \text{ otherwise}
        \end{cases}
    \]
    Hence, we have that 
    \[
        \left( \sum_{i \in T} \frac{\partial}{\partial X_{ii}} \right)^{|T|}p_{\vec{a}}(X) = \sum_{S \in \binom{[n]}{k} : T \subseteq S}
        a_S \det(X|_{S \setminus T}).
    \]

    Next, we will need to use the following two facts about Schur complements: 
    firstly, there is the Schur complement lemma, which states that
    \[
        \det(X) = \det(X|_T) \det(X \setminus T).
    \]
    Secondly, we have that Schur complements commute with taking submatrices, in that if $T \subseteq S \subseteq [n]$, then
    \[
        (X|_S) \setminus T = (X \setminus T)|_{S \setminus T}.
    \]

    Therefore, we see that 
    \begin{align*}
        \det(X|_T)\left( \sum_{i \in T} \frac{\partial}{\partial X_{ii}} \right)^{|T|}p_{\vec{a}}(X \setminus T)
        &= \sum_{S \in \binom{[n]}{k} : T \subseteq S}
        a_S \det(X|_T)\det((X\setminus T)|_{S \setminus T})\\
        &= \sum_{S \in \binom{[n]}{k} : T \subseteq S} a_S \det(X|_S)\\
        &= p_{\vec{a}, T}(X)
    \end{align*}

    
\end{proof}


\subsection{Weighted Determinantal Point Processes}
We now let $\vec{a} = (a_S : S \in \binom{[n]}{k}) \in \R_+^{\binom{n}{k}}$, so that these coefficients are assumed to be nonnegative.
Let $M$ be a fixed positive semidefinite definite matrix.

We define the weighted determinantal point process for this set of coefficients to be a proability distribution on $S \in \binom{[n]}{k}$ with probability mass function given by
\[
    \mu_{M, \vec{a}}(S) \propto a_S \det(M|_S).
\]

To relate this to sparse linear regression, we can imagine using the following randomized variant of sparse linear regression.
Given the data $A$ and $b$, first draw a random subset $S$ according to $\mu_{AA^{\intercal}, \vec{a}}$, and then perform the regression using only the columns of $A$ in $S$.

We can then notate the expected error produced by this randomized linear regression.
\[
    \eta_{\vec{a}, T, k}(A,b) = \E_{\mu_{AA^{\intercal},\vec{a}}} [\ell(A_S, b) | T \subseteq S].
\]
Once again, this expectation serves as an upper bound on the sparse linear regression problem.

We can then relate this quantity $\eta_{\vec{a}, T, k}(A, b)$ to the lpm polynomial $p_{\vec{a}}$:
\begin{theorem}
    If $AA^{\intercal}$ has rank at least $k$, then
\[
     \eta_{\vec{a}, T, k}(A,b) =
     \|b\|^2 - \frac{p_{\vec{a}, T}(A^{\intercal}A + A^{\intercal}bb^{\intercal}A)}{p_{\vec{a}, T}(AA^{\intercal})} + 1
 \]
\end{theorem}
\begin{proof}
    We first recall the so-called matrix determinant lemma \cite{TODO}, which states that for any invertible $X \in \R^{n\times n }$ and $v \in \R^{n}$
    \[
        \det(X + vv^{\intercal}) = (1+v^{\intercal}X^{-1}v)\det(X).
    \]

    This implies that
    \begin{align*}
        p_{\vec{a}, T}(A^{\intercal}A + A^{\intercal}bb^{\intercal}A) &= \sum_{S \in \binom{[n]}{k}, T \subseteq S} a_S\det(A^{\intercal}A|_S + A^{\intercal}bb^{\intercal}A|_S)\\
                                                                   &= \sum_{S \in \binom{[n]}{k}, T \subseteq S} a_S\det(A^{\intercal}A|_S)\big(1+b^{\intercal}A_S(A^{\intercal}A|_S)^{-1}A_S^{\intercal}b\big)\\
    \end{align*}

    We also recall the closed form formula for $\ell(A,b)$ \cite{TODO}, given by
    \[
        \ell(A,b) = \|b\|^2 - b^{\intercal}A(A^{\intercal}A)^{-1}A^{\intercal}b.
    \]

    We can thus simplify the above expression and see that

    \begin{align*}
        p_{\vec{a}, T}(A^{\intercal}A + A^{\intercal}bb^{\intercal}A) &= \sum_{S \in \binom{[n]}{k}, T \subseteq S} a_S\det(A^{\intercal}A|_S)\big(1+\|b\|^2-\ell(A_S,b)\big)\\
                                                                   &= (1+\|b\|^2)\left( \sum_{S \in \binom{[n]}{k}, T \subseteq S} a_S\det(A^{\intercal}A) \right) + \sum_{S \in \binom{[n]}{k}}\det(A)\ell(A|_S,b)\\
                                                                   &= (1+\|b\|^2)p_{\vec{a}}(A^{\intercal}A) + \sum_{S \in \binom{[n]}{k}, T \subseteq S}\det(A^{\intercal}A|_S)\ell(A_S,b)\\
    \end{align*}

    We then have that

    \begin{align*}
        \frac{p_{\vec{a}, T}(A^{\intercal}A + A^{\intercal}bb^{\intercal}A)}{p_{\vec{a}, T}(A^{\intercal}A)} &= (1+\|b\|^2) + \big(\sum_{S \in \binom{[n]}{k}, T \subseteq S}\Pr(S)\ell(A_S,b)\big)\\
                                                                                                      &= \|b\|^2 - 
                                                                                                      \eta_{\vec{a}, T, k}(A, b) + 1
    \end{align*}

    Rearranging, we obtain the result.
    
\rend{proof}
\begin{remark}
    There is a rich literature concerning the proiblem of efficiently sampling from truncated determinantal point processes \cite{TODO}, but it is worth noting that sampling based approaches will not be effective at computing this quantity to the accuracy that we wish to compute it.

   For instance, consider the unweighted case where $\vec{a} = \vec{1}$, and suppose that $A^{\intercal}A$ is such that for any $S \subseteq [n]$ with $|S| = k$, $\det(A^{\intercal}A|_S) = 1$.
   In this case, we see that the determinantal point process selects elements from $\binom{[n]}{k}$ uniformly at random.
   So, we can consider two cases: one in which $b$ is chosen to lie in the image of $A_S$, and one in which $b$ is chosen to be orthogonal to the image of $A$.
   If we attempted to distinguish these two cases using sampling, we would need to take $\Omega(\binom{[n]}{k})$ samples before we could find that there is some $S$ so that $\ell(A_S, b) < \|b\|^2$.
   On the other hand, by computing the expectation in these two cases, we could immediately detect which situation we are in.
\end{remark}

We have thus reduced the problem of computing $\eta_{\vec{a}, T, k}(A, b)$ to that of computing $p_{\vec{a}, T}$, which the following lemma makes formal.

\begin{lemma}
    If we can compute $p_{\vec{a}}(X)$ in $O(\tau)$ time for any complex symmetric matrix $X$, then we can compute  $\eta_{\vec{a}, T, k}(A, b)$ in $O(k\tau + k\log(k) + n^\omega)$ time, where $\omega$ denotes the matrix multiplication constant.
\end{lemma}
\begin{proof}
    We have seen that in order to compute $\eta_{\vec{a}, T, k}(A, b)$, it suffices to compute $p_{\vec{a}, T}$ at $A^{\intercal}A$ and $A^{\intercal}A+A^{\intercal}bb^{\intercal}A$.

    We have seen in lemma \ref{lem:poly_cond} that
    \[
        p_{\vec{a}, T}(X) = \det(X|_T) \left( \sum_{i \in T} \frac{\partial}{\partial X_{ii}} \right)^{|T|}p_{\vec{a}}(X \setminus T) 
    \]
    For symmetric matrices $X$, We can easily compute $X \setminus T$ and $\det(X|_T)$ in $n^{\omega}$ time, and so it remains to argue that we can compute 
    \[
         \left( \sum_{i \in T} \frac{\partial}{\partial X_{ii}} \right)^{|T|}p_{\vec{a}}(X)
    \]
    in the desired time.

    If we let $D$ be the diagonal matrix so that $D_{ii} = \begin{cases} 1 \text{ if }i \in T\\0 \text{ otherwise}\end{cases}$, we can see that we can rewrite this expression as a directional derivative:
    \[
        \left( \sum_{i \in T} \frac{\partial}{\partial X_{ii}} \right)^{|T|}p_{\vec{a}}(X) = \frac{d^{|T|}}{dt^{|T|}} p_{\vec{a}}(X + tD)|_{t = 0}. 
    \]
    Notice that $p_{\vec{a}}(X + tD)$ is a univariate polynomial in $t$ of degree $k$.
    Therefore, if we can compute this function at $k$ distinct values of $t$, then we can uniquely recover all of the coefficients of this univariate polynomial via interpolation.
    Computing this at any $k$ distinct values only requires $k\tau$ time.

    In fact, using the Fast-Fourier transform \cite{TODO}, it is in fact possible to reconstruct all of the coefficients of this polnymial this $k\log(k)$ time from its evaluations it at all $2^{\lceil \log(k) \rceil}$ roots of unity.

    Once we have all of the coefficients of $p_{\vec{a}}(X + tD)$, we can compute $\frac{d^{|T|}}{dt^{|T|}}$ in constant time by reading of the $|T|^{th}$ coefficient of this polynomial.
    This gives the final time complexity given in the theorem.
\end{proof}

\subsection{Characteristic Coefficients}
We will see that we can in fact obtain much faster methods when the polynomial in question is the characteristic coefficient defined above.

In fact, there are a number of algorithms for computing the characteristic coefficients of a symmetric matrix.
Here, we will use the the Fadeev-LeVerrier method for computing the characteristic coefficients, which has the advantage of not needing to compute all of the characteristic coefficients at once.
\begin{lemma}
    We can compute  $\eta_{\vec{1}, T, k}(A, b)$ in $O(kn^{\omega})$ time where $\omega$ is the matrix multiplication constant.
\end{lemma}
\begin{proof}
We will once again make use of our result that
\[
    p_{\vec{1}, T}(X) = \det(X|_T) \left( \sum_{i \in T} \frac{\partial}{\partial X_{ii}} \right)^{|T|}p_{\vec{1}}(X \setminus T) 
\]
We will note that for the characteristic coefficients,
\[
    \left( \sum_{i \in T} \frac{\partial}{\partial X_{ii}} \right)^{|T|}c_k^n(X)  = c_{k-|T|}^{n-|T|}(X|_{[n] \setminus T})
\]
Therefore, 
\[
    p_{\vec{1}, T}(X) = \det(X|_T) c_k^n(X \setminus T) 
\]

The Fadeev-LeVerrier algorithm computes $c_k^n$ in $O(kn^{\omega})$ time \cite{}, and the result follows.
\end{proof}

\subsection{A Heuristic for Sparse Regression}
We will focus this section on the unweighted determinantal point process, as that illustrates all of the relevant ideas, and is what we will use in our simulations.


We can now present our heuristic for finding a sparse regressor.
\begin{algorithm}
    \caption{The $\eta$-greedy method}
    \begin{algorithmic}\label{alg:greedy}
        \State $T \gets \varnothing$
        \For{$t = 1 \dots k$}
            \State $j \gets \argmin \eta_{T+j, k}(A,b)$
            \State $T \gets T + j$
        \EndFor

        \Return T
    \end{algorithmic}
\end{algorithm}
We can think of method as follows: we will iteratively construct the set which we will use for regression.
At each step, we will choose the element from $[n]$ which minimizes our expected error, if we condition on taking the elements we have already selected.
