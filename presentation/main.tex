\documentclass{beamer}

\usetheme[progressbar=frametitle]{metropolis}
\usepackage{appendixnumberbeamer}

\usepackage{booktabs}
\usepackage[scale=2]{ccicons}

\usepackage{pgfplots}
\usepgfplotslibrary{dateplot}

\usepackage{xspace}
\newcommand{\themename}{\textbf{\textsc{metropolis}}\xspace}

\DeclareMathOperator{\Sym}{Sym}
\usepackage{amsthm, amsmath, amsfonts, amssymb}
\usepackage{relsize}
\usepackage{graphicx}
\usepackage{subcaption}
\usepackage{bm}
\usepackage[english]{babel}
\usepackage{booktabs}
\usepackage{mathtools}
\usepackage{algorithm}
\usepackage{algpseudocode}
\usepackage[parfill]{parskip}

%Define theorem formatting

\newcommand{\R}{\mathbb{R}}
\newcommand{\C}{\mathbb{C}}
\newcommand{\N}{\mathbb{N}}
\newcommand{\Z}{\mathbb{Z}}
\newcommand{\Q}{\mathbb{Q}}
\newcommand{\E}{\mathbb{E}}
\newcommand{\Hom}{\textbf{\text{Hom}}}
\newcommand{\PP}{\textbf{P}}
\newcommand{\SPACE}{\textbf{SPACE}}
\newcommand{\NP}{\textbf{NP}}
\newcommand{\POS}{\mathcal{P}}
\newcommand{\SAT}{\textbf{SAT}}
\newcommand{\pard}[2]{\frac{\partial #1}{\partial #2}}
\DeclareMathOperator*{\argmin}{arg\,min}
\DeclareMathOperator*{\argmax}{arg\,max}
\DeclareMathOperator*{\diag}{diag}
\DeclareMathOperator*{\supp}{supp}
\DeclareMathOperator{\SDP}{SDP}
\DeclareMathOperator{\CUT}{MAXCUT}
\DeclareMathOperator{\conv}{\operatorname{conv}}
\DeclareMathOperator{\FW}{FW}
\DeclareMathOperator{\tr}{tr}
\DeclareMathOperator{\rank}{rank}
\newcommand{\st}{{\text{ s.t. }}}
\renewcommand\top[2]{\genfrac{}{}{0pt}{}{#1}{#2}}
\newcommand\twoline[2]{\genfrac{}{}{0pt}{}{#1}{#2}}

\colorlet{shadecolor}{gray!70}
\setbeamercolor{block body}{bg=shadecolor!30,fg=black}


\title{Quadratic Programs with Sparsity Constraints}
\author{Kevin Shu\inst{1}}
\institute{\inst{1} Georgia Institute of Technology}
\date{}

\begin{document}
\frame{\titlepage}
\begin{frame}
    \centering
    \huge
    {\color{gray}Sparse Quadratic Programming}
\end{frame}
\begin{frame}
\frametitle{Sparse Quadratic Programming}
    \begin{equation}
        \begin{aligned}
            \max\quad & x^{\intercal}A_0x\\
            \st & x^{\intercal}A_1x = 1\\
                &|\supp(x)| \le k
        \end{aligned}
    \end{equation}
    \vspace{0.125in}

    $A_0, A_1$ are symmetric, and $x \in \R^n$. 
    \[
        \supp(x) = \{i : x_i \neq 0\}.
    \]

\end{frame}
\begin{frame}
    \frametitle{Sparse PCA}
    \begin{equation}
        \begin{aligned}
            \max\quad & x^{\intercal}Ax\\
            \st & x^{\intercal}x = 1\\
                &|\supp(x)| \le k.
        \end{aligned}
    \end{equation}
\end{frame}
\begin{frame}
    \frametitle{Sparse Linear Regression}
    \begin{equation}
        \begin{aligned}
        \min & \|Ax - b\|_2^2\\
        \st & |\supp(x)| \le k.
        \end{aligned}
    \end{equation}
\end{frame}
\begin{frame}
    \frametitle{Sparse Linear Regression}
    \begin{figure}[h]
        \centering
        \includegraphics[width=\linewidth]{slide3.jpg}
    \end{figure}
\end{frame}
\begin{frame}
    \frametitle{Sparse Linear Regression}
    Sparse regression is a sparse QCQP where
    \[
        A_0 = A^{\intercal}bb^{\intercal}A
    \]
    \[
        A_1 = A^{\intercal}A
    \]
\end{frame}
\begin{frame}
    \centering
    \huge
    {\color{gray}A Convex Formulation}
\end{frame}
\begin{frame}
\frametitle{Semidefinite Programming Relaxations}
    For non-sparse QCQPs, there is a standard Schur relaxation, which is a convex
    optimization problem.
    \begin{equation}
        \begin{aligned}
            \max\quad & x^{\intercal}A_0x\\
            \st & x^{\intercal}A_1x = 1\\
        \end{aligned}
    \end{equation}
\end{frame}
\begin{frame}
\frametitle{Semidefinite Programming Relaxations}
    For non-sparse QCQPs, there is a standard Schur relaxation, which is a convex
    optimization problem.
    \begin{equation}
        \begin{aligned}
            \max\quad & \tr(A_0xx^{\intercal})\\
            \st & \tr(A_1xx^{\intercal}) = 1\\
        \end{aligned}
    \end{equation}
\end{frame}
\begin{frame}
\frametitle{Semidefinite Programming Relaxations}
    For non-sparse QCQPs, there is a standard Schur relaxation, which is a convex
    optimization problem.
    \begin{equation}
        \begin{aligned}
            \max\quad & \tr(A_0X)\\
            \st & \tr(A_1X) = 1\\
                & X \in S_n\\
        \end{aligned}
    \end{equation}
    \pause 
    \[
        S_n = \conv \{ xx^{\intercal} : x \in \R^n\} = \{X : X \succeq 0\}
    \]
\end{frame}
\begin{frame}
\frametitle{Semidefinite Programming Relaxations for Sparse QCQPs}
    For \emph{sparse} QCQPs, we can consider a sparse version of an SDP:
    \begin{equation}
        \begin{aligned}
            \max\quad & x^{\intercal}A_0x\\
            \st & x^{\intercal}A_1x = 1\\
                &|\supp(x)| \le k
        \end{aligned}
    \end{equation}
\end{frame}
\begin{frame}
\frametitle{Semidefinite Programming Relaxations for Sparse QCQPs}
    For \emph{sparse} QCQPs, we can consider a sparse version of an SDP:
    \begin{equation}
        \begin{aligned}
            \max\quad & \tr(A_0X)\\
            \st & \tr(A_1X) = 1\\
                & X \in \FW^k_n\\
        \end{aligned}
    \end{equation}
    \pause 
    \[
        \FW^k_n = \conv \{xx^{\intercal} : x\in \R^n,\;|\supp(x)| \le k\}.
    \]
\end{frame}
\begin{frame}
    \frametitle{Duality}
    There is a conical dual to $\FW^k_n$:
    \[
        S^k_n = \{Y \in \Sym : \forall X \in \FW^k_n,\;\langle X, Y \rangle \ge 0\}.
    \]
\end{frame}
\begin{frame}
    \frametitle{Duality}
    We can give the dual cone more explicitly:
    \[
        S^k_n = \{Y \in \Sym : \forall S : |S| = k, \;Y|_S \succeq 0\}.
    \]
\end{frame}
\begin{frame}
    \frametitle{Duality}
    For most values of $A_1$, the following are equivalent:
    \begin{columns}
        \begin{column}{0.5\textwidth}
            \begin{equation}
                \begin{aligned}
                    \max\quad & \tr(A_0X)\\
                    \st & \tr(A_1X) = 1\\
                        & X \in \FW^k_n\\
                \end{aligned}
            \end{equation}
        \end{column}
        \begin{column}{0.5\textwidth}
            \begin{equation}
                \begin{aligned}
                    \min\quad & y\\
                    \st & A_1y - A_0 \in S^k_n\\
                \end{aligned}
            \end{equation}
        \end{column}
    \end{columns}
\end{frame}
\begin{frame}
    \frametitle{Distances}
    \begin{block}{Question}
        Let $Y \in S^k_n$. How far away can $Y$ be from being PSD?
    \end{block}
    \pause
    \begin{block}{Question}
        Let $Y \in S^k_n$, and $\tr(Y)=1$. How small can $\lambda_{min}(Y)$ be?
    \end{block}
\end{frame}
\begin{frame}
    \centering
    \huge
    {\color{gray}Eigenvalues in $S^k_n$}
\end{frame}
\begin{frame}
    \frametitle{Eigenvalues in $S^k_n$}
    \begin{equation}
        \begin{aligned}
            \min\quad & \lambda_{min}(Y)\\
            \st & \tr(Y) = 1\\
                & Y \in S^k_n\\
        \end{aligned}
    \end{equation}
    \pause
    \begin{itemize}
        \item Concave minimization (hard).
        \pause
        \item Complicated objective.
    \end{itemize}
\end{frame}
\begin{frame}
    \frametitle{Eigenvalues in $S^k_n$}
    \begin{block}{Observation}
        $Y \in S^k_n$ if and only if

        \hspace{0.25in} \emph{for any $S \subseteq [n]$ with $|S| \le k$, $\det(Y|_S) \ge 0$.}
    \end{block}
    \pause 
    Define 
    \[
        c_n^k(Y) = \sum_{\top{S \subseteq [n]}{|S| = k}}  \det(Y|_S).
    \]
\end{frame}
\begin{frame}
    \frametitle{Eigenvalues in $S^k_n$}
    \begin{block}{A Note on Characteristic Coefficients}
        $c_n^k(Y)$ is sometimes called a \emph{characteristic coefficient}, since it is a coefficient of the characteristic polynomial of $Y$, i.e.
        \[
            \det(Y + tI) = \sum_{j=0}^n c_n^j(Y) t^j.
        \]
        \pause
        In particular, it is
        \begin{itemize}
            \item Basis invariant, so that $c_n^k(U^{\intercal} Y U) = c_n^k(Y)$ for any orthogonal matrix $U$.
            \pause
            \item Efficiently computable.
            \pause
            \item Hyperbolic... (undefined)
        \end{itemize}
    \end{block}
\end{frame}
\begin{frame}
    \frametitle{Eigenvalues in $S^k_n$}
    \begin{block}{Observation}
        If $Y \in S^k_n$, then for $j = 1 ,\dots, k$, $c_n^j(Y) \ge 0$.
    \end{block}
\end{frame}
\begin{frame}
    \frametitle{Eigenvalues in $S^k_n$}
    Relax the following program
    \begin{equation}
        \begin{aligned}
            \min\quad & \lambda_{min}(Y)\\
            \st & \tr(Y) = 1\\
                & Y \in S^k_n\\
        \end{aligned}
    \end{equation}
\end{frame}
\begin{frame}
    \frametitle{Eigenvalues in $S^k_n$}
    Relax the following program
    \begin{equation}
        \begin{aligned}
            \min\quad & \lambda_{min}(Y)\\
            \st & \tr(Y) = 1\\
                & {\color{red}\text{For }j=1 ,\dots, k,\; c_n^j(Y) \ge 0}\\
        \end{aligned}
    \end{equation}
\end{frame}
\begin{frame}
    \frametitle{Eigenvalues in $S^k_n$}
    Consider
    \[
        H = \{Y : \text{For }j=1 ,\dots, k,\; c_n^j(Y) \ge 0\}
    \]
    \pause This is
    \begin{itemize}
        \item Basis invariant, so that $U^{\intercal} Y U \in H$ if and only if $Y \in H$ for any orthogonal matrix $U$.
        \pause
        \item Efficiently computable.
        \pause
        \item Convex, which follows from the theory of \emph{hyperbolic polynomials}. It is called the \emph{hyperbolicity cone} associated with the polynomial $c_n^k$.
    \end{itemize}
\end{frame}
\begin{frame}
    \frametitle{Eigenvalues in $S^k_n$}
    \begin{equation}
        \begin{aligned}
            \min\quad & \lambda_{min}(Y)\\
            \st & \tr(Y) = 1\\
                & Y \in H
        \end{aligned}
    \end{equation}
\end{frame}
\begin{frame}
    \frametitle{Eigenvalues in $S^k_n$}
    \begin{equation}
        \begin{aligned}
            \min\quad & {\color{red}Y_{11}}\\
            \st & \tr(Y) = 1\\
                & {\color{red}Y \text{ is diagonal}}\\
                & Y \in H
        \end{aligned}
    \end{equation}
\end{frame}
\begin{frame}
    \frametitle{Eigenvalues in $S^k_n$}
    It turns out that this is a convex optimization problem, and it is symmetric up to permutations of the last $n-1$ coordinates, so we can actually solve this exactly.
    
    An optimum is given by a diagonal matrix $Y$ so that 
    \[
        Y_{11} = \frac{k-n}{n(k-1)}
    \]
    \[
        Y_{22} = Y_{33}  =\dots= Y_{nn} = \frac{(n+1)k}{n^2(k-1)}
    \]
\end{frame}
\begin{frame}
    \frametitle{Eigenvalues in $S^k_n$}
    This is also the diagonalization of a matrix
    \[
        G(n,k) = 
        \frac{1}{n}
        \begin{pmatrix}
            1 & -\frac{1}{k-1} & -\frac{1}{k-1}  &\dots& -\frac{1}{k-1}\\
            -\frac{1}{k-1} & 1 & -\frac{1}{k-1}  &\dots& -\frac{1}{k-1}\\
            \dots\\
            -\frac{1}{k-1} & -\frac{1}{k-1}& -\frac{1}{k-1}  &\dots & 1\\
        \end{pmatrix},
    \]
    which is in $S^k_n$.

    This implies that $\frac{k-n}{n(k-1)}$ is precisely the solution to the original optimization problem over $S^k_n$.
\end{frame}
\begin{frame}
    \frametitle{Takeaways}
    \begin{itemize}
        \item We can precisely compute the value of this nonconvex eigenvalue minimization problem.
        \item We can control the set $S^k_n$ (which is useful in sparse optimization) using polynomial functions (which are tractable).
    \end{itemize}
\end{frame}
\begin{frame}
    \frametitle{Stepping Back}
    %TODO: Include a picture.
\end{frame}
\begin{frame}
    \frametitle{Hyperbolicity Cone Relaxations}
    We can now perform a relaxation
    \begin{equation}
        \begin{aligned}
            \min\quad & y\\
            \st & A_1y - A_0 \in S^k_n\\
        \end{aligned}
    \end{equation}
\end{frame}
\begin{frame}
    \frametitle{Hyperbolicity Cone Relaxation}
    We can now perform a relaxation
        \begin{equation}
            \begin{aligned}
                \min\quad & y\\
                \st & A_1y - A_0 \in {\color{red} H(c_n^k)}\\
            \end{aligned}
        \end{equation}
    \pause
\end{frame}
\begin{frame}
    \frametitle{Hyperbolicity Cone Relaxation}
    %TODO: Add a picture
\end{frame}
\begin{frame}
    \frametitle{Hyperbolicity Cone Relaxation}
    \begin{block}{Key Fact}
        \[\partial H(c_n^k) = \{X \in H(c_n^k) : c_n^k(X) = 0\}.\]
    \end{block}
\end{frame}
\begin{frame}
    \frametitle{Hyperbolicity Cone Relaxation}
    So, the optimization problem 

    \begin{equation}
        \begin{aligned}
            \min\quad & y\\
            \st & A_1y - A_0 \in H(c_n^k)\\
        \end{aligned}
    \end{equation}

    is intimately related to the question, \emph{what are the roots of the univariate polynomial $g(y) = c_n^k(A_1y-A_0)$?}
\end{frame}
\begin{frame}
    \frametitle{Hyperbolicity Cone Relaxation}
    \begin{block}{Theorem}
        If $A_1$ is PSD, then 
        \begin{equation}
            \begin{aligned}
                \min\quad & y\\
                \st & A_1y - A_0 \in H(c_n^k)\\
            \end{aligned}
        \end{equation}
        is exactly the maximum root of $c_n^k(A_1y-A_0)$.
    \end{block}
\end{frame}
\begin{frame}
    \frametitle{Hyperbolicity Cone Relaxation}
    %TODO: Draw a picture.
\end{frame}
\begin{frame}
    \begin{block}{Theorem}
        \begin{equation}
            \begin{aligned}
                \max\quad & x^{\intercal}A_0x\\
                \st & x^{\intercal}A_1x = 1\\
                    &|\supp(x)| \le k.
            \end{aligned}
        \end{equation}
        is at least 
        \[
            \eta = \max \{y : c_n^k(A_1y-A_0) = 0 \}.
        \]
    \end{block}
\end{frame}
\begin{frame}
    \frametitle{Example}
    Remember sparse linear regression:
    \begin{equation}
        \begin{aligned}
        \max & x^{\intercal}A^{\intercal}bb^{\intercal}Ax\\
        \st & x^{\intercal}A^{\intercal}Ax = 1\\
            & |\supp(x)| \le k.
        \end{aligned}
    \end{equation}
\end{frame}
\begin{frame}
    \frametitle{Example}
    Plugging this into our expression for sparse linear regression, we want to compute
    \[
        \eta = \max \{y : c_n^k(A^{\intercal}Ay-\hat{b}\hat{b}^{\intercal}) = 0 \}.
    \]
    where $\hat{b} = A^{\intercal}b$.
\end{frame}
\begin{frame}
    \frametitle{Example}
    \[
        \hat{b}\hat{b}^{\intercal}
    \]
    is rank 1! That means $y = 0$ is a root of multiplicity $k-1$ of this polynomial, so actually
    \[
        c_n^k(A^{\intercal}Ay-\hat{b}\hat{b}^{\intercal}) = y^{k-1}(c_1y+c_2),
    \]
    for some $c_1,c_2 \in \R$.
\end{frame}
\begin{frame}
    \frametitle{Example}
    \[
        \eta = \frac{c_n^k(A^{\intercal}A+\hat{b}\hat{b}^{\intercal})}{c_n^k(A^{\intercal}A)} - 1.
    \]
\end{frame}
\begin{frame}
    \frametitle{Example}
    Sanity check: when $k = n$, this becomes
    \[
        \eta = \frac{\det(A^{\intercal}A+\hat{b}\hat{b}^{\intercal})}{\det(A^{\intercal}A)} - 1.
    \]
    \pause
    This turns out to be exactly the formula for the optimal error in ordinary least squares regression, and can be seen as a version of Cramer's rule for solving a system of linear equations.
\end{frame}
\begin{frame}
    \frametitle{Example}
    Two questions:
    \begin{itemize}
        \item Can we efficiently find a solution to the sparse QCQP that does at least as well as this result guarantees?
        \pause
        \item Is this formula any good?
    \end{itemize}
\end{frame}
\begin{frame}
    \frametitle{Example}
    Two questions:
    \begin{itemize}
        \item Can we efficiently find a solution to the sparse QCQP that does at least as well as this result guarantees?
            \begin{itemize}
                \item Yes!
            \end{itemize}

        \item Is this formula any good?
            \begin{itemize}
                \item No! But we can improve.
            \end{itemize}
    \end{itemize}
    Both of these can be done using the same idea of introducing coefficients.
\end{frame}
\begin{frame}
    \centering
    \huge
    {\color{gray}Introducing Coefficients}
\end{frame}
\begin{frame}
    Recall
    \[
        c_n^k(Y) = \sum_{\top{S \subseteq [n]}{|S| = k}}  \det(Y|_S).
    \]
\end{frame}
\begin{frame}
    A polynomial is said to be a linear principal minors polynomial (LPM) if there are $a_S$ so that
    \[
        p(Y) = \sum_{\top{S \subseteq [n]}{|S| = k}} a_S \det(Y|_S).
    \]
\end{frame}
\begin{frame}
    For any LPM polynomial $p$ with nonnegative coefficients, and any $Y \in S^k_n$, 
    \[
        p(Y) \ge 0.
    \]
    \pause
    So, if $A_1$ is PSD, then 
    \[
        \eta_p = \max \{y : p(A_1y-A_0) = 0 \}.
    \]
    is a lower bound on the value of the dual problem.
\end{frame}
\begin{frame}
    We can be a little more refined: let $p$ be LPM, and let 
    \[
        \supp(p) = \{S : a_S \neq 0\}.
    \]
    \pause
    Then,
    \begin{block}{Theorem}
        Let $p$ be an LPM polynomial with nonnegative coefficients.
        Consider the structured sparse QCQP:
        \begin{equation}
            \begin{aligned}
                \max\quad & x^{\intercal}A_0x\\
                \st & x^{\intercal}A_1x = 1\\
                    & \supp(x) \in \supp(p)
            \end{aligned},
        \end{equation}
        then there is some feasible $x$ acheiving an objective of at least $\eta_p$.
    \end{block}
\end{frame}
\begin{frame}
    Also, if $p = \det(X|_S)$, for some $S$, then $\eta_p$ is exactly
    \begin{equation}
        \begin{aligned}
            \max\quad & x^{\intercal}A_0x\\
            \st & x^{\intercal}A_1x = 1\\
                & \supp(x) = S
        \end{aligned}.
    \end{equation}
\end{frame}
\begin{frame}
    An idea: start with $p_0 = c_n^k$, and then construct polynomials $p_1,p_2 ,\dots, $ so that $\supp(p_{i+1}) \subsetneq \supp(p_{i})$.

    Hopes:
    \begin{itemize}
        \item At each step, $\eta_{p_i}$ is increasing.
        \item At the end, $p_k$ is supported on a single set.
        \pause
        \item $p_k$ can be efficiently evaluated!
    \end{itemize}
\end{frame}
\begin{frame}
    \frametitle{Conditioning}
    For a LPM polynomial $p$, and a set $T \subseteq [n]$, define
    \[
        p|_T = \sum_{S : T \subseteq S} a_S \det(X|_S).
    \]
    This has support strictly smaller than $p$. Is it better than $p$?
\end{frame}
\begin{frame}
    \frametitle{Conditioning}
    \begin{theorem}
    For any LPM polynomial $p$ with nonnegative coefficients, there exists some $i$ so that
    \[
        \eta_{p|_i} \ge \eta_{p}.
    \]
    \end{theorem}
\end{frame}
\begin{frame}
    \frametitle{Conditioning}
    Consider
        \begin{align*} 
        \sum_{i \in [n]} p|_{i}&=\sum_{i \in [n]} \sum_{i \in  S} a_S \det(X|_S)\\
                &=k\sum_{S} a_S \det(X|_S)\\
                &=kp
        \end{align*}
\end{frame}
\begin{frame}
    \frametitle{Conditioning}
    \[
        \sum_{i\in [n]}p|_i(\eta_p) = kp(\eta_p) = 0.
    \]
    \pause
    Therefore, for some $i$, $ p|_{i}(\eta_{p}) \le 0$.
    \pause
    This implies that the largest root of $p|_{i}$ is at least $\eta_{p}$.
\end{frame}
\begin{frame}
    \begin{algorithm}[H]
    \frametitle{An Algorithm for Sparse QCQPs}
    \caption{The Greedy Conditioning Heuristic}
    \label{alg:greedy}
    \begin{algorithmic}
        \State $T \gets \varnothing$
        \For{$t = 1 \dots k$}
            \State $j \gets \argmax \eta_{p|_{T + j}}$
            \State $T \gets T + j$
        \EndFor

        \Return T
    \end{algorithmic}
    \end{algorithm}
\end{frame}
\begin{frame}
    \frametitle{An Algorithm for Sparse QCQPs}
    
This always returns some $T$ so that 
\begin{equation}
    \begin{aligned}
        \max\quad & x^{\intercal}A_0x\\
        \st & x^{\intercal}A_1x = 1\\
            & \supp(x) = T
    \end{aligned}
\end{equation}
is at least $\eta_p$.
\end{frame}
\begin{frame}
    \frametitle{How do we efficiently compute $p|_T$?}
    Define the Schur-complement of $X$ with respect to $T \subseteq [n]$ to be
    \[
        X \setminus T = X - X_{T,:} X_{T,T}^{-1} X_{T,:}.
    \]

    Then,
    \[
        p|_T(X) = \det(X|_T) (\prod_{i\in  T} \frac{d}{dX_{ii}}) p(X \setminus T).
    \]
    In particular, this can be computed efficiently.
\end{frame}
%\begin{frame}
%    \frametitle{Positive Semidefinite Matrices}
%    \begin{block}{Definition}
%        An $n\times n$ symmetric matrix $A$ is PSD if all of its eigenvalues are nonnegative.
%    \end{block}
%
%    {\large \textbf{Equivalent Conditions}}
%    \begin{itemize}
%        \item The quadratic form $q(x) = x^{\intercal}Ax$ is nonnegative for all $x$.
%        \item The determinants of all principal submatrices of $A$ are nonnegative.
%        \item $A$ can be factored as $V^{\intercal}V$.
%        \item There are vectors $v_1 ,\dots, v_n$ so that $A_{ij} = \langle v_i, v_j\rangle$ for each $i$, $j$ in $[n]$.
%    \end{itemize}
%
%\end{frame}
%\begin{frame}
%    \frametitle{Positive Semidefinite Matrices}
%    \begin{figure}[H]
%    \begin{subfigure}{0.5\textwidth}
%        \[
%            \begin{pmatrix}
%                1 & 1 & 1 & 1\\
%                1 & 1 & 1 & 1\\
%                1 & 1 & 1 & 1\\
%                1 & 1 & 1 & 1
%            \end{pmatrix}
%        \]
%    \end{subfigure}%
%    \begin{subfigure}{0.5\textwidth}
%        \[
%            \begin{pmatrix}
%                1 & 2 & 3 & 4\\
%                2 & 3 & 4 & 5\\
%                3 & 4 & 5 & 6\\
%                4 & 5 & 6 & 7
%            \end{pmatrix}
%        \]
%    \end{subfigure}%
%    \end{figure}
%    \pause
%    \begin{figure}[H]
%    \begin{subfigure}{0.5\textwidth}
%        \vspace{0.1in}
%        This is PSD: it can be factored
%        \[
%            \begin{pmatrix}
%                1 \\ 1 \\ 1 \\ 1
%            \end{pmatrix}
%            \begin{pmatrix}
%                1 & 1 & 1 & 1
%            \end{pmatrix}.
%        \]
%
%    \end{subfigure}%
%    \pause
%    \begin{subfigure}{0.5\textwidth}
%        This is not PSD, the submatrix 
%        \[
%            \begin{pmatrix}
%                1 & 2 \\
%                2 & 3 
%            \end{pmatrix}
%        \]
%        has determinant $-1$.
%    \end{subfigure}%
%    \end{figure}
%
%\end{frame}
%\begin{frame}
%    \frametitle{Positive Semidefinite Matrices}
%    \begin{block}{Eigenvalue convexity}
%        If $X$ and $Y$ are PSD, then $X+Y$ is PSD.
%    \end{block}
%    \begin{block}{Eigenvalue convexity}
%        If $X$ is PSD, then $\lambda X$ is PSD for $\lambda \ge 0$.
%    \end{block}
%    This makes the set of all PSD matrices a convex cone.
%\end{frame}
%
%\begin{frame}
%    \frametitle{Conic Optimization}
%    \begin{figure}[h]
%        \centering
%        \includegraphics[width=0.6\linewidth]{l2_cone.png}
%        \caption{A geometric visualization of the conic optimization.}%
%    \end{figure}
%\end{frame}
%\begin{frame}
%    \frametitle{Conic Optimization}
%    \begin{figure}[h]
%        \centering
%        \includegraphics[width=0.6\linewidth]{slice_l2cone.png}
%        \caption{A geometric visualization of the conic optimization.}%
%    \end{figure}
%\end{frame}
%\begin{frame}
%    \frametitle{Conic Optimization}
%    \begin{figure}[h]
%        \centering
%        \includegraphics[width=0.6\linewidth]{slice_l2.png}
%        \caption{A geometric visualization of the conic optimization.}%
%    \end{figure}
%\end{frame}
%\begin{frame}
%    \frametitle{Conic Optimization}
%    \begin{figure}[h]
%        \centering
%        \includegraphics[width=0.6\linewidth]{optimizer.png}
%        \caption{A geometric visualization of the conic optimization.}%
%    \end{figure}
%\end{frame}
%\begin{frame}
%    \frametitle{Semidefinite Programming}
%    \begin{equation*}
%    \begin{aligned}
%        \text{minimize} &&\langle B^0, X\rangle\\
%        \text{such that } &&\langle B^{\ell}, X\rangle = b_{\ell} &&\text{ for }\ell \in \{1,\dots, k\}\\
%                          &&  X \succeq 0
%    \end{aligned}
%    \end{equation*}
%    \begin{itemize}
%        \item $X \succeq 0$ means that $X$ is an $n\times n $ positive semidefinite matrix.
%        \item A matrix is positive semidefinite if it is symmetric and all of its eigenvalues are nonnegative.
%        \item The $B^i$ are all $n\times n $ symmetric matrices.
%    \end{itemize}
%\end{frame}
%\begin{frame}
%    \frametitle{Interesting Semidefinite Programs}
%    
%    \begin{block}{Goemans-Williamson SDP}
%    For a graph $G$, with adjacency matrix $A$,
%    \begin{equation*}
%        \SDP_{GW} = 
%    \begin{aligned}
%        \text{minimize} &&\langle A, X\rangle\\
%        \text{such that } &&X_{ii} = 1 &\text{ for each }i\in [n]\\
%                          &&  X \succeq 0
%    \end{aligned}
%    \end{equation*}
%    \end{block}
%    \begin{itemize}
%        \item Let $\alpha = \frac{1}{2}(|E(G)| - \SDP_{GW}).$
%        \item If $\CUT$ is the size of the maximum number of edges in any bipartite subgraph of $G$, then
%            \[ 0.8781\alpha \le \CUT \le \alpha.\]
%    \end{itemize}
%\end{frame}
%\begin{frame}
%    \textbf{Example: }
%    If $G = C_4$ is the 4-cycle 
%
%    \begin{figure}[H]
%    \begin{subfigure}{0.35\textwidth}
%      \centering
%        \begin{tikzpicture}
%        % define the points of a regular pentagon
%            \node[circle,fill=red,minimum size=0.25mm,label={\small 3}](v1) at (0:1) {};
%            \node[circle,fill=black,minimum size=0.25mm,label={\small 4}](v2) at (90:1) {};
%            \node[circle,fill=red,minimum size=0.25mm,label={\small 1}](v3) at (180:1){};
%            \node[circle,fill=black,minimum size=0.25mm,label={\small 2}](v4) at (270:1){};
%            \node[circle,fill=red,minimum size=0.25mm](v5) at (360:1){};
%            \draw (v1) -- (v2) -- (v3) -- (v4) -- (v5) -- cycle;
%        \end{tikzpicture}
%        \caption{The 4-cycle, and an optimal bipartition of the vertices.}
%    \end{subfigure}%
%    \begin{subfigure}{0.65\textwidth}
%        \begin{equation*}
%        \begin{aligned}
%            \text{minimize} && X_{12}+X_{23}+X_{34}+X_{14}\\
%            \text{such that } &&\begin{pmatrix}
%                                 1& X_{12}&X_{13}&X_{14}\\
%                                  X_{12}& 1&X_{23}&X_{24}\\
%                                  X_{13}& X_{23}&1&X_{34}\\
%                                  X_{14}& X_{24}&X_{34}& 1
%                              \end{pmatrix}
%                                  \succeq 0
%        \end{aligned}
%        \end{equation*}
%    \end{subfigure}
%    \caption{Goemans Williamson SDP for the 4-cycle. This does turn out to give the correct MAX-CUT value of $4$.}
%    \end{figure}
%\end{frame}
% \begin{frame}
%     \frametitle{Interesting Semidefinite Programs}
%     \begin{block}{Lovasz Theta Number}
%     For a graph $G$, 
%     \begin{equation*}
%         \SDP_{\theta} = 
%     \begin{aligned}
%         \text{minimize} &&\langle \vec{1}, X\rangle\\
%         \text{such that } &&X_{ij} = 0 &\text{ if }i,j\in E(G)\\
%                           && \tr(X) = 1\\
%                           &&  X \succeq 0
%     \end{aligned}
%     \end{equation*}
%     \end{block}
%     \begin{itemize}
%         \item If $\omega(G)$ is the clique number of $G$, and $\chi(G)$ is the chromatic number. $\bar{G}$ is the complement graph to $G$.
%         \item $\omega(\bar{G}) \le \SDP_{\theta} \le \chi(\bar{G})$.
%     \end{itemize}
% \end{frame}
%\begin{frame}
%    Other applications to industrial engineering, control theory and more!
%
%    Semidefinite programs let you express complicated constraints efficiently.
%\end{frame}
%\begin{frame}
%    \centering
%    \huge
%    {\color{gray}Sparse Semidefinite Programming}
%\end{frame}
%\begin{frame}
%    \frametitle{Solving Semidefinite Programs}
%    \begin{itemize}
%        \item Interior point methods make semidefinite programs efficiently$^*$ solvable!
%        \pause
%        \item Memory costs of solving semidefinite programs are often high in practice, especially when we need to compute Hessians.
%        \pause
%        \item Graphs with 1000 vertices turn into semidefinite programs with 500,000 variables!
%        \pause
%        \item How can we improve performance costs of solving semidefinite programs?
%    \end{itemize}
%\end{frame}
%\begin{frame}
%    \frametitle{Sparsity}
%    Our notion of sparsity will always be parameterized by a graph, $G$.
%
%    \begin{block}{Definition}
%        A semidefinite program is \textbf{$G$-sparse} if it does not use the variables $X_{ij}$ when $i,j \not \in E(G)$ in the linear constraints or objective.
%    \end{block}
%\end{frame}
%\begin{frame}
%    \textbf{Example: }Goemans-Williamson semidefinite programs are $G$-sparse.
%    If $G = C_4$ is the 4-cycle 
%
%    \begin{figure}[H]
%    \begin{subfigure}{0.35\textwidth}
%      \centering
%        \begin{tikzpicture}
%        % define the points of a regular pentagon
%            \node[circle,fill=black,minimum size=0.25mm,label={\small 3}](v1) at (0:1) {};
%            \node[circle,fill=black,minimum size=0.25mm,label={\small 4}](v2) at (90:1) {};
%            \node[circle,fill=black,minimum size=0.25mm,label={\small 1}](v3) at (180:1){};
%            \node[circle,fill=black,minimum size=0.25mm,label={\small 2}](v4) at (270:1){};
%            \node[circle,fill=black,minimum size=0.25mm](v5) at (360:1){};
%            \draw (v1) -- (v2) -- (v3) -- (v4) -- (v5) -- cycle;
%        \end{tikzpicture}
%        \caption{The 4-cycle.}
%    \end{subfigure}%
%    \begin{subfigure}{0.65\textwidth}
%        \begin{equation*}
%        \begin{aligned}
%            \text{minimize} && X_{12}+X_{23}+X_{34}+X_{14}\\
%            \text{such that } &&X_{11}=X_{22}=X_{33}=X_{44}=1\\
%                              &&X \succeq 0
%        \end{aligned}
%        \end{equation*}
%    \end{subfigure}
%    \caption{Goemans Williamson SDP for the 4-cycle}
%    \end{figure}
%    \pause
%    \emph{I don't care about the values of $X_{13}$ and $X_{24}$, as long as they make the matrix PSD!}
%\end{frame}
%\begin{frame}
%    \frametitle{Sparsity}
%    We define the $G$-partial matrices to be symmetric matrices, where entries corresponding to nonedges of $G$ are `forgotten'.
%
%    \begin{block} {Definition}
%    A $G$-partial matrix is \textbf{PSD-completable} if the missing entries can be chosen to make the resulting symmetric matrix PSD.
%
%    $\Sigma(G)$ is the convex cone of PSD completable $G$-partial matrices.
%    \end{block}
%    \begin{figure}[h]
%    \[
%          \begin{pmatrix}
%              X_{11} & X_{12} & ? & X_{14}\\
%              X_{12} & X_{22} & X_{23} & ?\\
%              ? & X_{23} & X_{33} & X_{34}\\
%              X_{14} & ? & X_{34} & X_{44}\\
%          \end{pmatrix}
%    \]
%        \caption{A $C_4$-partial matrix.}%
%    \end{figure}
%\end{frame}
%\begin{frame}
%    \frametitle{Sparsity}
%    \begin{figure}[H]
%    \begin{subfigure}{0.35\textwidth}
%      \centering
%      \[
%          \begin{pmatrix}
%              1 & 1 & ? & ?\\
%              1 & 1 & 1 & ?\\
%              ? & 1 & 1 & 1\\
%              ? & ? & 1 & 1\\
%          \end{pmatrix}
%      \]
%
%        \caption{This partial matrix is PSD-completable by replacing all the ?'s by 1's.}
%    \end{subfigure}%
%    \begin{subfigure}{0.65\textwidth}
%      \[
%          {\color{red}
%          \begin{pmatrix}
%              1 & 2 & ? & ?\\
%              2 & 1 & 1 & ?\\
%              ? & 1 & 1 & 1\\
%              ? & ? & 1 & 1\\
%      \end{pmatrix}}
%      \]
%
%        \caption{This partial matrix is not PSD-completable.}
%    \end{subfigure}
%    \end{figure}
%\end{frame}
%\begin{frame}
%    \frametitle{Sparsity}
%    $G$-sparse SDP's can be thought of as conic optimization problems over a projection of the PSD cone.
%
%    For example
%    {\small
%    \begin{equation*}
%    \begin{aligned}
%        \text{minimize} && X_{12}+X_{23}+X_{34}+X_{14}\\
%        \text{such that } &&X_{11}=X_{22}=X_{33}=X_{44}=1\\
%                          &&X \succeq 0
%    \end{aligned}
%    \end{equation*}
%    }
%    can be rewritten
%    {\small
%    \begin{equation*}
%    \begin{aligned}
%        \text{minimize} && X_{12}+X_{23}+X_{34}+X_{14}\\
%        \text{such that } &&X_{11}=X_{22}=X_{33}=X_{44}=1\\
%                          &&
%                          \begin{pmatrix}
%                              X_{11} & X_{12} & ? & X_{14}\\
%                              X_{12} & X_{22} & X_{23} & ?\\
%                              ? & X_{23} & X_{33} & X_{34}\\
%                              X_{14} & ? & X_{34} & X_{44}\\
%                          \end{pmatrix} \in \Sigma(G)
%    \end{aligned}
%    \end{equation*}
%    }
%\end{frame}
%\begin{frame}
%    \frametitle{Sparsity}
%    Optimization of $G$-sparse SDP's is equivalent to the problem of linear optimization over slices of $\Sigma(G)$.
%
%    Can we do this optimization without using the full SDP?
%
%    \pause
%    \emph{Idea: if a matrix is PSD, then any principal submatrix of that matrix is PSD.}
%
%    \[
%          \begin{pmatrix}
%              \color{red}1 & \color{red}0 & 0 & 0\\
%              \color{red}0 & \color{red}1 & 0 & 0\\
%              0 & 0 & 1 & 0\\
%              0 & 0 & 0 & 1\\
%          \end{pmatrix}
%    \]
%
%\end{frame}
%\begin{frame}
%    \frametitle{A Natural Relaxation}
%    \begin{block}{Definition}
%        A $G$-partial matrix is \textbf{$G$-locally PSD} if all of its fully specified prinicipal submatrices are PSD.
%
%        $\POS(G)$ is the convex cone of $G$-locally PSD matrices.
%    \end{block}
%
%    Fully specified prinicipal submatrices are in correspondence with cliques of $G$.
%    \begin{figure}[h]
%        \begin{subfigure}{0.4\textwidth}
%            \begin{tikzpicture}
%            % define the points of a regular pentagon
%                \node[circle,fill=red,minimum size=0.25mm](v1) at (0:1) {};
%                \node[circle,fill=red,minimum size=0.25mm](v2) at (90:1) {};
%                \node[circle,fill=red,minimum size=0.25mm](v3) at (180:1){};
%                \node[circle,fill=black,minimum size=0.25mm](v4) at (270:1){};
%                \node[circle,fill=black,minimum size=0.25mm](v5) at (360:1){};
%                \draw (v1) -- (v2) -- (v3) -- (v4) -- (v5) -- cycle;
%                \draw[color=red] (v3) -- (v2);
%            \end{tikzpicture}
%        \end{subfigure}
%        \begin{subfigure}{0.4\textwidth}
%            \[
%                  \begin{pmatrix}
%                      \color{red}X_{11} & \color{red}X_{12} & ? & X_{14}\\
%                      \color{red}X_{12} & \color{red}X_{22} & X_{23} & ?\\
%                      ? & X_{23} & X_{33} & X_{34}\\
%                      X_{14} & ? & X_{34} & X_{44}\\
%                  \end{pmatrix}
%            \]
%        \end{subfigure}
%    \end{figure}
%\end{frame}
%\begin{frame}
%    \frametitle{A Natural Relaxation}
%    If we have a $G$-sparse SDP, 
%    \begin{equation*}
%        \SDP =
%    \begin{aligned}
%        \text{minimize} &&\langle B^0, X\rangle\\
%        \text{such that } &&\langle B^{\ell}, X\rangle = b_{\ell} &&\text{ for }\ell \in \{1,\dots, k\}\\
%                          &&  X \in \Sigma(G),
%    \end{aligned}
%    \end{equation*}
%    We will denote a modification 
%    \begin{equation*}
%        \SDP^{SG} = 
%    \begin{aligned}
%        \text{minimize} &&\langle B^0, X\rangle\\
%        \text{such that } &&\langle B^{\ell}, X\rangle = b_{\ell} &&\text{ for }\ell \in \{1,\dots, k\}\\
%                          &&  X \in \mathcal{P}(G)
%    \end{aligned}
%    \end{equation*}
%\end{frame}
%\begin{frame}
%    \frametitle{A Natural Relaxation}
%
%    \textbf{Advantages}
%    \begin{itemize}
%        \item Smaller PSD conditions are easier to check than larger PSD conditions.
%        \item We don't need to consider variables that aren't in $E(G)$.
%    \end{itemize}
%    \pause
%
%    \textbf{Disadvantages}
%    \begin{itemize}
%        \item Some graphs have exponentially many cliques
%        \item The approximation need not be good.
%    \end{itemize}
%\end{frame}
%\begin{frame}
%    \frametitle{Chordal Graphs and Equality}
%    It is natural to ask when the above relaxation is exactly equal, i.e. when $\Sigma(G) = \POS(G)$.
%
%    This was shown by Grone, Johnson, S\'{a}, and Wolkowicz.
%    \begin{block}{Definition}
%        $G$ is chordal if it has no induced cycles with more than 3 vertices.
%    \end{block}
%    \begin{block}{Theorem}
%        $\Sigma(G) = \POS(G)$ if and only if $G$ is chordal.
%    \end{block}
%\end{frame}
%\begin{frame}
%    \frametitle{Chordal Graphs and Equality}
%    Given a graph $G$, and a $G$-sparse SDP, it is standard practice to find a new graph $G'$ that is chordal and contains $G$.
%
%    We can then think of a $G$-sparse SDP as a $G'$ sparse SDP, and then use the above theorem to optimize over $\POS(G)$ instead of $\Sigma(G)$.
%
%    \textbf{Disadvantages}
%    \begin{itemize}
%        \item Computing a chordal graph containing $G$ with minimal number of edges is NP-hard (there are $O(\log(n))$-factor approximations though).
%        \item Chordal graphs containing $G$ might contain a lot more edges than $G$.
%    \end{itemize}
%\end{frame}
%\begin{frame}
%    \centering
%    \huge
%    {\color{gray}Approximate Semidefinite Programming}
%\end{frame}
%\begin{frame}
%    \textbf{Key Question:} How well does $\POS(G)$ approximate $\Sigma(G)$?
%
%    For chordal $G$, these are equal.
%
%    In experimental settings, it is often seen that optimizing over $\POS(G)$ is almost equivalent to optimizing over $\Sigma(G)$, even when $G$ is not chordal.
%    How can we quantify this?
%\end{frame}
%\begin{frame}
%    \begin{figure}[h]
%        \centering
%        \includegraphics[width=0.6\linewidth]{expansion.png}
%        \caption{A geometric visualization of $\epsilon(G)$.}%
%    \end{figure}
%\end{frame}
%\begin{frame}
%    \textbf{Approximate PSDness}
%
%    For any $G$, we let $I_G$ be the projection of the identity onto the $G$-partial matrices.
%
%    \begin{block} {Definition}
%    If $X$ is a $G$-partial matrix, then $\lambda(X)$ is the largest $\lambda$ so that
%    \[
%        X - \lambda I_G \in \Sigma(G).
%    \]
%    \end{block}
%
%    \begin{itemize}
%        \item $\lambda(X)$ is the largest possible value of the minimum eigenvalue of $\hat{X}$, where $\hat{X}$ is a completion of $X$.
%
%        \item $\lambda(X) \ge 0$ if and only if $X \in \Sigma(G)$.
%        
%    \end{itemize}
%\end{frame}
%\begin{frame}
%    \[
%        \lambda\left(
%          \begin{pmatrix}
%              1 & 1 & ? & -1\\
%              1 & 1 & 1 & ?\\
%              ? & 1 & 1 & 1\\
%              -1 & ? & 1 & 1
%      \end{pmatrix}  \right) = 1 - \sqrt{2}.
%    \]
%    $\lambda(X)$ is the minimum eigenvalue of the matrix
%    \[
%          \begin{pmatrix}
%              1 & 1 & 0 & -1\\
%              1 & 1 & 1 & 0\\
%              0 & 1 & 1 & 1\\
%              -1 & 0 & 1 & 1
%          \end{pmatrix}.
%    \]
%\end{frame}
%\begin{frame}
%    
%\end{frame}
%\begin{frame}
%    $\lambda(X)$ is a concave function of $X$; its minimum occurs at an extreme point.
%
%    The extreme points look like
%\end{frame}
%\begin{frame}
%    \begin{block} {Definition}
%    For any $G$,
%    \[
%        \epsilon(G) = \max \{-\lambda(X) : X \in \POS(G),\;\tr(X) = 1\}.
%    \]
%    \end{block}
%    This is `how far from being PSD completable a matrix in $\POS(G)$ can be'.
%
%    For any $X \in \POS(G)$, $X + \epsilon(G) \tr(X) I_G$ is PSD completable.
%\end{frame}
%\begin{frame}
%    \begin{figure}[h]
%        \centering
%        \includegraphics[width=0.6\linewidth]{expansion.png}
%        \caption{A geometric visualization of $\epsilon(G)$.}%
%    \end{figure}
%    $\epsilon(G)$ is the smallest number so that 
%    \[
%        \tilde{\Sigma}(G) \subseteq \tilde{\POS}(G) \subseteq (1+n \epsilon(G))\tilde{ \Sigma}(G).
%    \]
%
%\end{frame}
%\begin{frame}
%    \begin{block}{Definition}
%        A SDP is said to be of \textbf{Goemans-Williamson Type} if 
%        \begin{itemize}
%            \item Every feasible point satisfies $\tr(X) \le n$.
%            \item $I_G$ is a feasible point.
%            \item The trace of the objective is 0.
%            \item It is a maximization problem.
%        \end{itemize}
%    \end{block}
%\end{frame}
%\begin{frame}
%    \begin{block}{Theorem}
%        Let $\SDP$ be some $G$-sparse semidefinite program, and $\SDP^{SG}$ be its relaxation.
%
%        If $\alpha$ is the value of $\SDP$, and $\alpha'$ is the value of $\SDP^{SG}$, then
%        \[
%            \alpha \le \alpha' \le (1+n\epsilon(G)) \alpha.
%        \]
%
%    \end{block}
%\end{frame}
%\begin{frame}
%    \textbf{Advantages}
%    \begin{itemize}
%        \item As long as $\epsilon(G)$ is $o(\frac{1}{n})$, we get a good approximation.
%        \item Don't need chordal supergraphs, as long as we can enumerate the cliques of $G$.
%    \end{itemize}
%    \pause
%    \textbf{Disadvantages}
%    \begin{itemize}
%        \item The approximation is not always good..
%        \item Computing $\epsilon(G)$ is a concave minimization problem, which tend to be difficult.
%    \end{itemize}
%
%    What graphs have $\epsilon(G) = o(\frac{1}{n})$?
%\end{frame}
%\begin{frame}
%    \centering
%    \huge
%    {\color{gray}Computing $\epsilon(G)$}
%\end{frame}
%\begin{frame}
%    For cycles, we have 
%    \begin{block} {Theorem}
%        \[
%            \epsilon(C_n) = \frac{1}{n}\left(\frac{1}{\cos(\frac{\pi}{n})}-1\right) =\theta(\frac{1}{n^3}).
%        \]
%    \end{block}
%    This is the $-\lambda(X)$ where
%    \[
%        X = 
%        \begin{pmatrix}
%            1 & 1 & ? & ?  &\dots& -1\\
%            1 & 1 & 1 & ?  &\dots& ?\\
%            ? & 1 & 1 & 1  &\dots& ?\\
%            &&\dots\\
%            -1 & ? & ? & ?  &\dots& 1
%        \end{pmatrix}.
%    \]
%
%    Key idea is the cycle conditions, and the fact that they are convex in certain parameters.
%\end{frame}
%\begin{frame}
%    \begin{block}{Definition}
%        The chordal girth of $G$ is the smallest number of vertices in an induced cycle of $G$ with at least 3 vertices, and $\infty$ if $G$ is chordal.
%
%        We denote this by $\gamma(G)$.
%    \end{block}
%    \begin{block} {Corollary}
%        If $G$ is series parallel, then
%        \[
%            \epsilon(G) = \epsilon(C_{\gamma(G)}) = \theta(\frac{1}{\gamma(G)^3}).
%        \]
%    \end{block}
%\end{frame}
%\begin{frame}
%    \begin{block}{Theorem}
%        If $G$ and $H$ are graphs, and $K \subseteq G$ and $K \subseteq H$ are cliques, then we denote by  $G \oplus H$ the \textbf{clique sum} of $G$ and $H$.
%
%        \[
%            \epsilon(G \oplus H) = \max\{\epsilon(G), \epsilon(H)\}.
%        \]
%
%    \end{block}
%    \begin{figure}[h]
%        \centering
%        \includegraphics[width=0.8\linewidth]{cliquesum.png}
%        \caption{Clique sums are obtained by gluing together two graphs along a clique.}%
%        \label{fig:cliquesum}
%    \end{figure}
%
%\end{frame}
%
%\begin{frame}
%    \begin{block}{Theorem}
%        If $G$ is a graph, let $\hat{G}$ denote the \textbf{cone} over $G$, then
%
%        \[
%            \epsilon(\hat{G}) = \epsilon(G).
%        \]
%
%    \end{block}
%    \begin{figure}[h]
%        \centering
%        \includegraphics[width=0.4\linewidth]{cone.png}
%        \caption{A cone of a graph is a graph that adds a single new vertex to $G$ connected to all vertices of $G$.}
%    \end{figure}
%
%\end{frame}
%\begin{frame}
%    \frametitle{Proof Idea for cycles}
%    $\lambda(X)$ is concave, so it is minimized at an extreme point.
%
%    For the cycle, the extreme points of $\mathcal{P}(G)$ all look like
%    \[
%        \begin{pmatrix}
%            a_1^2 & a_1a_2  &\dots& \pm a_1a_n\\
%            a_1a_2 & a_2^2  &\dots& ?\\
%            \dots\\
%            \pm a_1a_n & ?  &\dots& a_n^2
%        \end{pmatrix} \sim
%        \begin{pmatrix}
%            a_1 \\ a_2  \\\dots\\ a_n 
%        \end{pmatrix}
%        \begin{pmatrix}
%            a_1 & a_2  &\dots& a_n 
%        \end{pmatrix}
%    \]
%    It is `almost rank 1', but one of the signs of the entries is wrong.
%\end{frame}
%\begin{frame}
%    \frametitle{Proof Idea for cycles}
%    We'll use the characterization that PSD matrices come from inner products of vectors.
%
%    Given an extreme point $X$, we look for a collection of vectors so that
%    \[
%        X_{ij} \sim \langle v_i, v_j\rangle = \|v_i\|\|v_j\|\cos(\theta).
%    \]
%    We'll make the guess that the $v_i$ should lie in $\R^2$.
%
%    This is a `highschool geometry problem'; we want to find some vectors on the unit circle so that their angles satisfy certain properties.
%\end{frame}
%\begin{frame}
%    \frametitle{Proof Idea for cycles}
%    The angle from $v_i$ to $v_{i+1}$ `should be' $\arccos(\frac{X_{ii+1}}{\sqrt{X_{ii}X_{i+1i+1}}})$. But if these angles come from vectors in $\R^2$, then we should have
%    \[
%        2\pi - \sum_{i=1}^{n}\arccos\left(\frac{X_{ii+1}}{\sqrt{X_{ii}X_{i+1i+1}}}\right) = 0
%    \]
%    If $X$ is not PSD completable, then this need not hold, and the `angle defect' increases as $\lambda(X)$ decreases.
%
%    It turns out that this angle defect is convex$^*$, and so it is maximized
%
%\end{frame}
%\begin{frame}
%    \centering
%    \huge
%    {\color{gray}Thickened Graphs}
%\end{frame}
%\begin{frame}
%    \begin{block}{Definition}
%        Given a graph $G$, a \textbf{thickening of }$G$ is obtained by replacing the edges of $G$ by chordal graphs with marked endpoints.
%    \end{block}
%    \begin{figure}[H]
%        \centering
%        \resizebox{3in}{1.5in}{
%        \begin{tikzpicture}[
%            mainnode/.style={circle, fill=black, minimum size=4mm},
%            smallnode/.style={circle, fill=black, minimum size=1mm},
%            ]
%            \node[mainnode] (1) at (0,0) {}; % Top Right
%            \node[mainnode] (2) at (0,3) {}; % Top left
%            \node[mainnode] (3) at (3,0) {}; % Bottom Right
%            \node[mainnode] (4) at (3,3) {}; % Bottom Left
%
%            \draw [-](2) to [out=50, in=130,looseness=15] (2);
%            \path [-](1) edge node[left] {} (2);
%            \path [-](1) edge node[left] {} (3);
%            \path [-](2) edge node[left] {} (4);
%            \path [-](3) edge [bend left] node[left] {} (4);
%            \path [-](3) edge [bend right] node[left] {} (4);
%
%            \coordinate (5) at (7,0);  % Bottom Left
%            \coordinate (6) at (7,3);  % Top left
%            \coordinate (7) at (10,0); % Bottom Right
%            \coordinate (8) at (10,3); % Top Right
%
%            \coordinate  (9) at (10.5,1.5); % Left Part of the right complex
%            \coordinate  (10) at (11,1.5); % Right part of the right complex
%
%            \coordinate (11) at (8.5,-1) {}; % Bottom clique
%            \coordinate (12) at (8.5,-0.4) {}; % Bottom clique
%
%            \coordinate (13) at (6.75,4) {}; % Loop
%            \coordinate (14) at (7.25,4) {}; % Loop
%            \coordinate (16) at (6.75,4.5) {}; % Loop
%            \coordinate (15) at (7.25,4.5) {}; % Loop
%
%            \filldraw[draw=black, fill=gray!20] (7) -- (9) -- (8) -- (10) -- cycle;
%            \filldraw[draw=black, fill=gray!50] (7) -- (5) -- (11) -- cycle;
%            \filldraw[draw=black, fill=gray!20] (6) -- (13) -- (14) -- (6) to [in = 20, out = 0] (15) -- (16) to [in = 160, out = 180] cycle;
%
%            \node[mainnode] (5n) at (7,0) {};  % Bottom Left
%            \node[mainnode] (6n) at (7,3) {};  % Top left
%            \node[mainnode] (7n) at (10,0) {}; % Bottom Right
%            \node[mainnode] (8n) at (10,3) {}; % Top Right
%
%            \node[smallnode] (9n) at (10.5,1.5) {}; % Left Part of the triangle
%            \node[smallnode] (10n) at (11,1.5) {}; % Right part of the triangle
%            \node[smallnode] (11n) at (8.5,-1) {}; % Bottom clique
%            \node[smallnode] (12n) at (8.5,-0.4) {}; % Bottom clique
%
%            \node[smallnode] (13n) at (6.75,4) {}; % Loop
%            \node[smallnode] (14n) at (7.25,4) {}; % Loop
%            \node[smallnode] (16n) at (6.75,4.5) {}; % Loop
%            \node[smallnode] (15n) at (7.25,4.5) {}; % Loop
%
%            \path [-](5) edge node[left] {} (6);
%            \path [-](5) edge node[left] {} (7);
%            \path [-](6) edge node[left] {} (8);
%            \path [-](7) edge [bend left] node[left] {} (8);
%
%            % Complex to the right
%            \path [-](7) edge node[left] {} (9);
%            \path [-](7) edge node[left] {} (10);
%            \path [-](8) edge node[left] {} (9);
%            \path [-](8) edge node[left] {} (10);
%            \path [-](9) edge node[left] {} (10);
%
%            % Clique
%            \path [-](11) edge node[left] {} (12);
%            \path [-](7) edge node[left] {} (12);
%            \path [-](5) edge node[left] {} (12);
%
%            % Loop
%            \path [-](13) edge node[left] {} (15);
%            \path [-](14) edge node[left] {} (15);
%            \path [-](13) edge node[left] {} (16);
%        \end{tikzpicture}
%        }
%        \caption{An example of a thickened graph. To the left, is a graph, and to the right is a thickening, where some of the edges have been replaced by other chordal graphs.}%
%        \label{fig:thickened_graph}
%    \end{figure}
%\end{frame}
%\begin{frame}
%    \begin{block}{Definition (Cont.)}
%        Formally, for each $e \in G$, we get some chordal graph $C_e$, and two distinguished vertices $a_e, b_e \in C_e$, and we glue $a_e$ to one of the endpoints of $e$, and $b_e$ to the other.
%    \end{block}
%\end{frame}
%\begin{frame}
%    \begin{block}{Theorem}
%        Suppose that $G$ is a thickened graph, and $e$ is any edge of $G$. Let $G / e$ be the contraction of $G$ along the edge $e$.
%
%        \[
%            \epsilon(G) \le \epsilon(G / e).
%        \]
%
%    \end{block}
%\end{frame}
%\begin{frame}
%    \frametitle{Completing to Thickened Graphs}
%    If $G$ is any graph, and we can break the graph down into pieces, and then find chordal covers of each piece separately to get a completion of $G$ to a thickened graph.
%    \begin{figure}[h]
%        \centering
%        \includegraphics[width=0.7\linewidth]{graph1.png}
%    \end{figure}
%\end{frame}
%\begin{frame}
%    \frametitle{Completing to Thickened Graphs}
%    If $G$ is any graph, and we can break the graph down into pieces, and then find chordal covers of each piece separately to get a completion of $G$ to a thickened graph.
%    \begin{figure}[h]
%        \centering
%        \includegraphics[width=0.7\linewidth]{graph2.png}
%    \end{figure}
%\end{frame}
%\begin{frame}
%    \frametitle{Completing to Thickened Graphs}
%    If $G$ is any graph, and we can break the graph down into pieces, and then find chordal covers of each piece separately to get a completion of $G$ to a thickened graph.
%    \begin{figure}[h]
%        \centering
%        \includegraphics[width=0.7\linewidth]{graph3.png}
%    \end{figure}
%\end{frame}
%\begin{frame}
%    \frametitle{Completing to Thickened Graphs}
%    If $G$ is any graph, and we can break the graph down into pieces, and then find chordal covers of each piece separately to get a completion of $G$ to a thickened graph.
%    \begin{figure}[h]
%        \centering
%        \includegraphics[width=0.7\linewidth]{graph4.png}
%    \end{figure}
%\end{frame}
%\begin{frame}
%    \begin{block}{Definition}
%        If $G$ is a graph, and $H$ is a thickening of $G$, then we denote by $H^p$ the subgraph of $H$ induced by the shortest paths from $a_e$ to $b_e$ in $C_e$ for each $e$.
%    \end{block}
%    \begin{figure}[H]
%        \centering
%        $H^p = $
%        \resizebox{1.5in}{1.5in}{
%        \begin{tikzpicture}[
%            mainnode/.style={circle, fill=black, minimum size=4mm},
%            smallnode/.style={circle, fill=black, minimum size=1mm},
%            ]
%
%            \coordinate (5) at (7,0);  % Bottom Left
%            \coordinate (6) at (7,3);  % Top left
%            \coordinate (7) at (10,0); % Bottom Right
%            \coordinate (8) at (10,3); % Top Right
%
%            \coordinate  (9) at (10.5,1.5); % Left Part of the right complex
%            \coordinate  (10) at (11,1.5); % Right part of the right complex
%
%            \coordinate (11) at (8.5,-1) {}; % Bottom clique
%            \coordinate (12) at (8.5,-0.4) {}; % Bottom clique
%
%            \coordinate (13) at (6.75,4) {}; % Loop
%            \coordinate (14) at (7.25,4) {}; % Loop
%            \coordinate (16) at (6.75,4.5) {}; % Loop
%            \coordinate (15) at (7.25,4.5) {}; % Loop
%
%            \filldraw[draw=black, fill=gray!20] (7) -- (9) -- (8) -- (10) -- cycle;
%            \filldraw[draw=black, fill=gray!50] (7) -- (5) -- (11) -- cycle;
%            \filldraw[draw=black, fill=gray!20] (6) -- (13) -- (14) -- (6) to [in = 20, out = 0] (15) -- (16) to [in = 160, out = 180] cycle;
%
%            \path [-, color=red](5) edge node[left] {} (6);
%            \path [-, color=red](5) edge node[left] {} (7);
%            \path [-, color=red](6) edge node[left] {} (8);
%            \path [-, color=red](7) edge [bend left] node[left] {} (8);
%
%            % Complex to the right
%            \path [-, color=red](7) edge node[left] {} (9);
%            \path [-](7) edge node[left] {} (10);
%            \path [-, color=red](8) edge node[left] {} (9);
%            \path [-](8) edge node[left] {} (10);
%            \path [-](9) edge node[left] {} (10);
%
%            % Clique
%            \path [-](11) edge node[left] {} (12);
%            \path [-](7) edge node[left] {} (12);
%            \path [-](5) edge node[left] {} (12);
%
%            % Loop
%            \path [-](13) edge node[left] {} (15);
%            \path [-](14) edge node[left] {} (15);
%            \path [-](13) edge node[left] {} (16);
%            \path [-, color=red](6) edge node[left] {} (14);
%            \path [-, color=red](6) edge node[left] {} (13);
%            \path [-, color=red](14) edge node[left] {} (13);
%
%            \node[mainnode, color=red] (5n) at (7,0) {};  % Bottom Left
%            \node[mainnode, color=red] (6n) at (7,3) {};  % Top left
%            \node[mainnode, color=red] (7n) at (10,0) {}; % Bottom Right
%            \node[mainnode, color=red] (8n) at (10,3) {}; % Top Right
%
%            \node[smallnode, color=red] (9n) at (10.5,1.5) {}; % Left Part of the triangle
%            \node[smallnode] (10n) at (11,1.5) {}; % Right part of the triangle
%            \node[smallnode] (11n) at (8.5,-1) {}; % Bottom clique
%            \node[smallnode] (12n) at (8.5,-0.4) {}; % Bottom clique
%
%            \node[smallnode, color=red] (13n) at (6.75,4) {}; % Loop
%            \node[smallnode, color=red] (14n) at (7.25,4) {}; % Loop
%            \node[smallnode] (16n) at (6.75,4.5) {}; % Loop
%            \node[smallnode] (15n) at (7.25,4.5) {}; % Loop
%
%        \end{tikzpicture}
%        }
%        \label{fig:thickened_graph}
%    \end{figure}
%\end{frame}
%\begin{frame}
%    \begin{block}{Theorem}
%        If $G$ is a graph, and $H$ is a thickening of $G$ obtained by replacing all of the edges of $G$ by paths of length $\ell$, then
%        \[
%            \epsilon(H) \le \epsilon(C_{\ell})
%        \]
%
%    \end{block}
%\end{frame}
%\begin{frame}
%    
%\end{frame}
%\begin{frame}
%    \bibstyle{plain}
%    \begin{thebibliography}{9}
%        \bibitem{preservers} Borcea, Julius, and Petter Brändén. "The Lee-Yang and Pólya-Schur programs. I. linear operators preserving stability." Inventiones mathematicae 177.3 (2009): 541.
%        \bibitem{interlacers} Kummer, Mario, Daniel Plaumann, and Cynthia Vinzant. "Hyperbolic polynomials, interlacers, and sums of squares." Mathematical Programming 153.1 (2015): 223-245.
%        \bibitem{positivity} Saunderson, James. "Certifying polynomial nonnegativity via hyperbolic optimization." SIAM Journal on Applied Algebra and Geometry 3.4 (2019): 661-690.
%    \end{thebibliography}
%\end{frame}
\end{document}
