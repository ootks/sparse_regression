\documentclass{amsart}
\usepackage[utf8]{inputenc}

\usepackage{amsthm}
\usepackage{amsmath}
\usepackage{amsfonts}
\usepackage{amssymb}
\usepackage{fullpage}
\usepackage{xcolor}
\usepackage{graphicx}
\usepackage{float}
\usepackage{algorithm}
\usepackage{algpseudocode}
\usepackage{tablefootnote}
\usepackage{caption}
\usepackage{subcaption}

\usepackage{hyperref,cleveref,color,verbatim}


\newtheorem{theorem}{Theorem}[section]
\newtheorem{prop}[theorem]{Proposition}
\newtheorem{lemma}[theorem]{Lemma}
\newtheorem{cor}[theorem]{Corollary}
\newtheorem{conj}[theorem]{Conjecture}
\newtheorem{question}[theorem]{Question}
\newtheorem{false}[theorem]{False Statement}
\newtheorem{observation}[theorem]{Observation}
\newtheorem*{theorem*}{Theorem}

\theoremstyle{definition}
\newtheorem{remark}[theorem]{Remark}
\newtheorem{defi}[theorem]{Definition}
\newtheorem{example}[theorem]{Example}

\newcommand{\fS}{\mathfrak{S}}
\newcommand{\R}{\mathbb{R}}
\newcommand{\Q}{\mathbb{Q}}
\newcommand{\C}{\mathbb{C}}
\newcommand{\EE}{\mathbb{E}}
\newcommand{\Z}{\mathbb{Z}}
\newcommand{\tr}{\textup{tr}}
\newcommand{\Ad}{\textup{Ad}}
\newcommand{\GL}{\operatorname{{\mathbf GL}}}
\newcommand{\gl}{\operatorname{gl}}
\DeclareMathOperator{\cond}{cond}
\newcommand{\FW}{\mathcal{F}\mathcal{W}}
\renewcommand{\P}{\mathcal{P}}
\newcommand{\M}{\mathcal{M}}

%%%%% debatable notations %%%%%
%%%%%%%%%%%%%%%%%%%%%%%%%%%%%%%
\newcommand\Def[1]{\emph{#1}}%
\newcommand\x{{x}}%
\newcommand\X{{X}}%
\newcommand\A{{A}}%
\newcommand\I{{I}}%
\renewcommand\a{{a}}
\renewcommand\v{{v}}%
%%%%%%%%%%%%%%%%%%%%%%%%%%%%%%%

\DeclareMathOperator*{\adj}{adj}
\DeclareMathOperator*{\Diag}{Diag}
\DeclareMathOperator*{\diag}{diag}
\DeclareMathOperator*{\Block}{B}
\DeclareMathOperator*{\Adj}{Adj}
\DeclareMathOperator*{\Span}{span}
\DeclareMathOperator*{\SO}{SO}
\DeclareMathOperator*{\supp}{supp}
\newcommand{\st}{{\text{ s.t. }}}
\newcommand{\transpose}{\intercal}
\DeclareMathOperator{\conv}{\operatorname{conv}}

\newcommand*{\Sym}{\R^{n \times n}_{\mathrm{sym}}}

\newcommand{\cL}{{\mathcal L}}
\newcommand{\De}{\operatorname{D}}
\newcommand\ks[1]{{\color{green}(Kevin: #1)}}


\DeclareMathOperator{\adju}{\operatorname{adj}}
\DeclareMathOperator{\argmin}{argmin}
\DeclareMathOperator{\argmax}{argmax}

\title{Minimizing Condition Number via Diagonal Congruence}
\date{\today}

\author{Kevin Shu}
\address[Kevin Shu]{School of Mathematics, Georgia Institute of Technology, 686 Cherry Street, Atlanta, GA 30332, USA}
\email{kshu8@gatech.edu}

\thanks{}

\begin{document}

\begin{abstract}
    We will consider a convex method for minimizing the condition number of a positive definite matrix by diagonal congruence.
\end{abstract}

\maketitle
The condition number of an $n\times n$ positive definite matrix $X$ is given by
\[
    \cond(X) = \frac{\lambda_{max}(X)}{\lambda_{min}(X)}.
\]
This measures how far $X$ is from being invertible, and is always at least 1.

One approach to preconditioning a matrix is to rescale it using diagonal congruence, that is for a diagonal matrix $D$, we want to consider 
\[
    \min_{D \text{ diagonal}}\cond(DXD).
\]
Diagonal congruence preserves many important properties of a matrix, but can improve the condition number significantly.
\begin{theorem}
    Let $X$ be a positive definite matrix. Consider the constrained convex minimization problem
    \begin{equation}
        \label{eq:opt1}
        \begin{aligned}
            \min\quad & t\\
            \st &tD - X \succeq 0\\
                &X - D \succeq 0\\
                &D \succ 0\\
                &D \text{ diagonal}.\\
        \end{aligned}
    \end{equation}
    Let $D^*$ be the diagonal matrix optimizing this program, and let $t^*$ be the optimal value of this program.
    \[
        \min_{D \text{ diagonal}}\cond(DXD) = t^* = \cond((D^*)^{-1/2}X(D^*)^{-1/2}).
    \]
\end{theorem}
\begin{proof}
    Let $(t^*,D^*)$ be the optimal values of this program.
    
    Notice that because $X - D^* \succeq 0$, by diagonal congruence,
    \[
        (D^*)^{-1/2}X(D^*)^{-1/2} \succeq I,
    \]
    so that in particular, $\lambda_{min}((D^*)^{-1/2}X(D^*)^{-1/2}) \ge 1$.
    
    We also have that for any fixed $D$, when $t$ is minimized, $tD - X$ must be singular, or else we could deccrease $t$ and remain in the feasible region.
    So, we see that
    \[
        t^*I \succeq (D^*)^{-1/2}X(D^*)^{-1/2},
    \]
    and that some eigenvalue of $(D^*)^{-1/2}X(D^*)^{-1/2}$ is $t$. From this, we obtain that $\lambda_{max}((D^*)^{-1/2}X(D^*)^{-1/2}) = t$.

    Therefore, $\cond((D^*)^{-1/2}X(D^*)^{-1/2}) \ge t^*$.

    On the other hand, let $D$ be any diagonal matrix minimizing $\cond(DXD)$. Because $\cond(DXD)$ is invariant to scaling $D$, we can assume $\lambda_{min}(DXD) = 1$.
    Now, let $D^* = D^{-2}$, and let $t^* = \lambda_{max}(DXD)$. Then, we have that $D \succeq 0$,
    \[
        X - D^* \succeq 0 \Leftrightarrow DXD \succeq I,
    \]
    and
    \[
        t^*D^* - X \succeq 0 \Leftrightarrow t^* I \succeq DXD.
    \]
    Therefore, $(t^*, D^*)$ is a feasible point for this program, with value $\cond(DXD)$, as desired.

\end{proof}
Notice that the problem defined in \Cref{eq:opt1} is not convex, but for fixed $t$, this program is a semidefinite programming feasibility problem. We can then optimize this program by performing binary search to find the smallest $t$ so that there exists a diagonal matrix $D$ so that $tD - X \succeq 0$, $X - D \succeq 0$ and $D \succ 0$.


We will also consider a related problem, of minimizing an analogue of condition number, namely 
\[
    \cond'(X) = \frac{\tr(X)}{\lambda_{min}(X)}.
\]
Clearly, we have that 
\[
    \cond(X) \le \cond'(X) \le n\cond'(X).
\]

We will see that $\cond'(X)$ can be formulated as a genuinely convex problem.
\begin{theorem}
    Let $X$ be a positive definite matrix. Consider the constrained convex minimization problem
    \begin{equation}
        \label{eq:opt}
        \begin{aligned}
            \min\quad & \tr(XD^{-1})\\
            \st & X - D \succeq 0\\
                &D \succ 0\\
                &D \text{ diagonal}.\\
        \end{aligned}
    \end{equation}
    Let $D^*$ be the diagonal matrix optimizing this program, and let $\kappa^*$ be the optimal value of this program.
    \[
        \min_{D \text{ diagonal}}\cond'(DXD) = \kappa^* = \cond'((D^*)^{-1/2}X(D^*)^{-1/2}).
    \]
\end{theorem}
\begin{proof}
    If $D^*$ is an optimizer for this quantity, we know that
    \[
        X - D^* \succeq 0.
    \]
    If $X - D^*$ is not singular, then clearly for some $\ell > 1$, we know that $\ell D^*$ is also feasible, and it will have a smaller objective value.
    So, we may assume that this matrix is singular.

    We can then perform a congruence transformation to see that
    \[
        (D^*)^{-1/2}X(D^*)^{-1/2} - I \succeq 0,
    \]
    and this matrix is also singular.

    Therefore, 
    \[
        \lambda_{min}((D^*)^{-1/2}X(D^*)^{-1/2}) = 1.
    \]
    From this, we see that 
    \begin{align*}
        \cond'((D^*)^{-1/2}X(D^*)^{-1/2}) &= \frac{\tr((D^*)^{-1/2}X(D^*)^{-1/2})}{\lambda_{min}((D^*)^{-1/2}X(D^*)^{-1/2}) } \\
        &= \tr((D^*)^{-1}X)\\
        &= \kappa^*.
    \end{align*}
    Here, we have used the cyclic property of the trace.

    Now, if $D$ is an invertible diagonal matrix, then there is some diagonal matrix $U$ so that $U$ has $\pm 1$ diagonal entries, and $DU$ is positive semidefinite.
    We see then That
    \[
        \cond'(DXD) = \cond'(UDXDU),
    \]
    so let us assume $D$ is positive semidefinite.

    Also notice that $\cond(DXD)$ is invariant to scaling $D$ by positive mutliples, so we may assume that $\lambda_{min}(DXD) = 1$.

    So, let $D^* = D^{-2}$. Then because $DXD \succeq I$, $X - D^* \succeq 0$. Therefore, $D^*$ is a feasible point, and
    \[
        \tr(XD^{-1}) = \tr(DXD) = \cond'(DXD).
    \]
    This gives the desired result.
\end{proof}
The function $\tr(XD^{-1})$ is strictly convex when $D$ is positive definite, so this is in fact a convex minimization problem.
\end{document}

