\documentclass[a4paper]{article}
\usepackage[margin=.8in]{geometry}
\usepackage{graphicx}
\usepackage{fancyhdr, multicol}
\usepackage{float}
\fancyhead[L]{\LARGE{}}
\fancyhead[C]{\LARGE{Sparse Linear Regression with $S^{n,k}$}}
\fancyhead[R]{\LARGE{}}

\newcommand{\FW}{\mathcal{F}\mathcal{W}}

\author{}
\title{Sparse Linear Regression with $S^{n,k}$}

%Set up fancy headers.
\pagestyle{fancy}
\usepackage{amsmath, amsthm, amsfonts, amssymb}
\usepackage[parfill]{parskip}

%Define theorem formatting
\newtheorem{theorem}{Theorem}
\newtheorem{lemma}{Lemma}

% Declare some common abbreviations
\newcommand{\R}{\mathbb{R}}
\newcommand{\C}{\mathbb{C}}
\newcommand{\N}{\mathbb{N}}
\newcommand{\Z}{\mathbb{Z}}
\newcommand{\Q}{\mathbb{Q}}
\newcommand{\E}{\mathbb{E}}
\newcommand{\Hom}{\textbf{\text{Hom}}}
\newcommand{\PP}{\textbf{P}}
\newcommand{\SPACE}{\textbf{SPACE}}
\newcommand{\NP}{\textbf{NP}}
\newcommand{\SAT}{\textbf{SAT}}
\newcommand{\pard}[2]{\frac{\partial #1}{\partial #2}}
\DeclareMathOperator*{\argmin}{arg\,min}
\DeclareMathOperator*{\argmax}{arg\,max}
\DeclareMathOperator*{\tr}{tr}
\newcommand{\st}{{\text{ s.t. }}}

\begin{document}
\maketitle

The sparse linear regression problem is to find
\[
    \min \{\|Ax - b\|^2 : x \in \R^n, \|x\|_0 \le k\},
\]
where $A \in \R^{n,m}$ and $b\in \R^m$, and $\|x\|_0$ denotes the number of nonzero entries of $x$. This is an NP-hard problem.

It is shown in \cite{subset} that this can be expressed as a convex optimization problem:
\begin{equation}\label{eq:cvx_prg}
\begin{aligned}
    \alpha = 
    \text{minimize} &&\|b\|^2 - \tr(XA^{\intercal}bb^{\intercal}A)\\
    \text{such that } && \tr(AA^{\intercal}X) = 1\\
                      && X \in \FW^{n,k}
\end{aligned}
\end{equation}
That is, the optimum of this program is of the form $\alpha xx^{\intercal}$ where $x$ is the optimal solution to the original sparse linear regression problem.

For my own sake, I'll include a proof that this works at the end in Lemma \ref{lem:convex}.

Let $H^{n,k}$ denote the hyperbolicity cone of the characteristic coefficient $c_{n,k}$, and let $C^{n,k}$ denote the dual of $H^{n,k}$.
Consider the modified program defined as 
\begin{equation}\label{eq:tight}
    \alpha_H = 
\begin{aligned}
    \text{minimize} &&\|b\|^2 - \tr(XA^{\intercal}bb^{\intercal}A)\\
    \text{such that } && \tr(AA^{\intercal}X) = 1\\
                      && X \in C^{n,k}
\end{aligned}.
\end{equation}
\begin{theorem}
    Let $\chi(AA^{\intercal})$ denote the condition number of $AA^{\intercal}$, then
   \[
       \alpha_H \ge \alpha \ge  (1+\frac{n-k}{(k-1)}\chi(AA^{\intercal}))\alpha_H.
   \]
\end{theorem}
\begin{proof}

We know that $C^{n,k} \subseteq \FW^{n,k}$, so that $\alpha_H \ge \alpha$.

We know that for any $X \in \FW^{n,k}$, $X + \frac{n-k}{(k-1)n}\tr(X)I \in C^{n,k}$. Moreover, if $X$ is a feasible point for Equation \ref{eq:cvx_prg}, then so is
\[
    \frac{1}{1+\frac{n-k}{(k-1)n}\tr(X)\tr(AA^{\intercal})}\left(X + \frac{n-k}{(k-1)n}\tr(X)I\right).
\]

We see then that this is a feasible point for Equation \ref{eq:tight}, whose value is 
\begin{align*}
    \|b\|^2 - \frac{1}{1+\frac{n-k}{(k-1)n}\tr(X)\tr(AA^{\intercal})}\tr\left(A^{\intercal}bb^{\intercal}A    \left(X + \frac{n-k}{(k-1)n}\tr(X)I\right)\right)\\
    = 
    \frac{1}{1+\frac{n-k}{(k-1)n}\tr(X)\tr(AA^{\intercal})}\left( \|b\|^2 - \tr\left(A^{\intercal}bb^{\intercal}A    X  \right)\right) + \\
        \frac{\frac{n-k}{(k-1)n}\tr(X)\tr(AA^{\intercal})}{1+\frac{n-k}{(k-1)n}\tr(X)\tr(AA^{\intercal})}\left( \|b\|^2 - \frac{\tr\left(A^{\intercal}bb^{\intercal}A\right)}{\tr(AA^{\intercal})}\right)\\
    = 
    \frac{\alpha}{1+\frac{n-k}{(k-1)n}\tr(X)\tr(AA^{\intercal})} + \frac{\frac{n-k}{(k-1)n}\tr(X)\tr(AA^{\intercal})}{1+\frac{n-k}{(k-1)n}\tr(X)\tr(AA^{\intercal})}\left( \|b\|^2 - \frac{\tr\left(A^{\intercal}bb^{\intercal}A\right)}{\tr(AA^{\intercal})}\right)\\
    \ge \alpha_H
\end{align*}
Notice that 
\[
    \frac{\frac{n-k}{(k-1)n}\tr(X)\tr(AA^{\intercal})}{1+\frac{n-k}{(k-1)n}\tr(X)\tr(AA^{\intercal})}\left( \|b\|^2 - \frac{\tr\left(A^{\intercal}bb^{\intercal}A\right)}{\tr(AA^{\intercal})}\right) \ge 0
\]

Therefore, in particular, 
\[
    (1+\frac{n-k}{(k-1)n}\tr(X)\tr(AA^{\intercal}))\alpha_H \le \alpha
\]

Finally, we note that $\tr(AA^{\intercal}X) = 1$, so $\lambda_{min}(AA^{\intercal})\tr(X) \le 1$, and therefore
\[
    (1+\frac{n-k}{(k-1)}\frac{\tr(AA^{\intercal})}{n\lambda_{min}(AA^{\intercal})})\alpha_H \le \alpha.
\]
Finally, notice that $\tr(AA^{\intercal}) \le n\lambda_{max}(A)$, so $\frac{\tr(AA^{\intercal})}{\lambda_{min}(AA^{\intercal})} \le n\chi(A)$, and we have shown the theorem.

\end{proof}



\begin{lemma}\label{lem:convex}
    The convex optimization problem in Equation \ref{eq:cvx_prg} solves the sparse linear regression problem.
\end{lemma}
\begin{proof}
    We can rewrite the objective above as
    \[
        \|Ax - b\|^2 = \tr(A^{\intercal}Axx^{\intercal}) - 2b^{\intercal}A^{\intercal}x + \|b\|^2
    \]

    Let $\lambda = \|x\|$, and let $y = \frac{x}{\lambda}$ then we have that 
    \[
        \|Ax - b\|^2 = \tr(A^{\intercal}Ay^{\intercal})\lambda^2 - 2b^{\intercal}Ay\lambda + \|b\|^2
    \]
    Minimizing this over $\lambda$ explicitly yields
    \[
        \min_{\|x\|_0 \le k} \|Ax - b\|^2 = \min_{y :\|y\|_0 \le k, \|y\| = 1}  \|b\|^2 - \frac{\tr(A^{\intercal}bb^{\intercal}Ayy^{\intercal})}{\tr(A^{\intercal}Ayy^{\intercal})}.
    \]
    We can normalize this by rescaling $y$ so that $\tr(A^{\intercal}Ayy^{\intercal} = 1$, resulting in the program
        
    \begin{equation}
    \begin{aligned}
        \text{minimize} &&\|b\|^2 - \tr(A^{\intercal}bb^{\intercal}Ayy^{\intercal})\\
        \text{such that } && \tr(AA^{\intercal}yy^{\intercal}) = 1\\
                          && \|y\|_0 \le k
    \end{aligned}
    \end{equation}

    Taking convex hulls yields that this program is equivalent to
        
    \begin{equation}
    \begin{aligned}
        \text{minimize} &&\|b\|^2 - \tr(A^{\intercal}bb^{\intercal}AX)\\
        \text{such that } && \tr(AA^{\intercal}X) = 1\\
                          && X \in \FW^k_n.
    \end{aligned}
    \end{equation}

\end{proof}
\subsection{What if there is a really good sparse regressor?}
The dual to the conic optimization problem
\begin{equation}
\begin{aligned}
    \text{minimize} &&\|b\|^2 - \tr(XA^{\intercal}bb^{\intercal}A)\\
    \text{such that } && \tr(AA^{\intercal}X) = 1\\
                      && X \in K
\end{aligned}
\end{equation}
is the problem
\begin{equation}
\begin{aligned}
    \text{maximize} &&\|b\|^2 - y\\
    \text{such that } && AA^{\intercal}y - A^{\intercal}bb^{\intercal}A  \in K^*\\
\end{aligned}
\end{equation}

\begin{thebibliography}
    
    \bibitem{subset} Walid Ben-Ameur, José Neto, New bounds for subset selection from conic relaxations, European Journal of Operational Research, Volume 298, Issue 2, 2022, Pages 425-438.

\end{thebibliography}

\end{document}
